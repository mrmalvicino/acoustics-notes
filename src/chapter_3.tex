\chapter{Ondas estacionarias}
\label{cha:standingWaves}

Una onda estacionaria es aquella que presenta puntos de amplitud nula que están fijos en el espacio, llamados nodos.
Resulta de la interferencia entre dos ondas de igual amplitud, longitud de onda y frecuencia pero que avanzan en sentido opuesto.

En la naturaleza este fenómeno puede darse, por ejemplo, en sogas para ondas transversales o en tubos para ondas longitudinales.
En estos casos, el modelo matemático usado es el de onda de una dimensión válido para frecuencias menores a:
\begin{equation*}
    f_c = \frac{1.84 \, c}{2\pi r}
\end{equation*}

Ya que para frecuencias mayores o para tubos de un radio $r>>\lambda$ pueden formarse ondas estacionarias transversales.
El comportamiento \emph{modal} de conductos de mayor radio o de recintos en general se analiza a partir de ondas tridimensionales.


\section{Resonancia en tubos}

Se dice que un tubo está en resonancia cuando uno de sus dos extremos está siendo excitado periódicamente por una perturbación de manera que se formen ondas estacionarias en el interior.

El extremo donde se da la perturbación periódica se modela mediante un pistón rígido que se mueve a una frecuencia angular $\omega$ hacia adelante y hacia atrás.
El otro extremo o bien puede estar abierto o bien puede ser cerrado por una tapa rígida.
En ambos casos, se genera una onda reflejada debido al cambio de impedancia.
Con lo cual, las funciones de presión y velocidad están dadas por las ecuaciones \ref{eqn:Helmholtz} y \ref{eqn:particleVel1D} respectivamente:
\begin{gather}
    p(x) = A \, e^{-\iu k x} + B \, e^{\iu k x}
    \label{eqn:tubePressure}
    \\
    v(x) = \frac{A}{\rho_0 \, c} \, e^{-\iu k x} - \frac{B}{\rho_0 \, c} \, e^{\iu k x}
    \label{eqn:tubeVelocity}
\end{gather}

De todas las frecuencias que pueda generar el pistón, solo van a ser resonantes aquellas para las que el tiempo en que tarde el pistón en realizar un ciclo esté sincronizado con el tiempo que tarda la onda en recorrer el tubo ida y vuelta
De esta forma, la onda reincide en el pistón cuando este está en la posición de equilibrio generando un nuevo frente de onda coincidente con el anterior.
En este extremo, la velocidad de las partículas va a ser determinada por la frecuencia impuesta por el pistón.
\begin{equation}
    v(x_0) = v_\omega
    \label{eqn:tubeV0}
\end{equation}

Consideremos el caso en que el otro extremo está cerrado por una tapa rígida, como se muestra en el esquema a continuación.

\begin{center}
    \def\svgwidth{0.6\linewidth}
    \input{./images/tubo-cerrado-cerrado.pdf_tex}
\end{center}

Dado que las partículas no van a poder moverse más allá de la tapa, la velocidad particular en ese punto será nula.
Por lo tanto, podemos asumir como hipótesis que en $x=L$ la impedancia es infinita.
\begin{equation*}
    \left\{
    \begin{aligned}
        p(L) &\to \infty
        \\
        v(L) &\to 0
    \end{aligned}
    \right.
    \implies
    Z \to \infty
\end{equation*}

A partir de estas hipótesis, evaluando $x=L$ en la ecuación \ref{eqn:tubeVelocity} y considerando la ecuación \ref{eqn:tubeV0} para $x=0$, se tiene que para cada extremo la velocidad de partículas es:
\begin{equation*}
    \left\{
    \begin{aligned}
        v(L) &= 0 = \frac{A}{\rho_0 \, c} e^{-\iu k L} - \frac{B}{\rho_0 \, c} e^{\iu k L}
        \\
        v(x_0) &= v_\omega = \frac{A}{\rho_0 \, c} - \frac{B}{\rho_0 \, c}
    \end{aligned}
    \right.
\end{equation*}

De este sistema se despeja $A = B \, e^{2 \iu kL}$ de la primer ecuación y se reemplaza en la segunda para obtener:
\begin{equation*}
    v_\omega \, \rho_0 \, c = B \, e^{2 \iu kL} - B
\end{equation*}

Y al despejar $B$:
\begin{align*}
    B &= \frac{v_\omega \, \rho_0 \, c}{e^{2 \iu kL}-1}
    \\
    &= \frac{v_\omega \, \rho_0 \, c \, e^{-\iu kL}}{e^{\iu kL}-e^{-\iu kL}}
    \\
    &= \frac{v_\omega \, \rho_0 \, c}{2\iu \sin(kL)} \, e^{-\iu kL}
\end{align*}

Se resuelve para $A$:
\begin{equation*}
    A = \frac{v_\omega \, \rho_0 \, c}{2\iu \sin(kL)} \, e^{\iu kL}
\end{equation*}

A partir de los coeficientes $A$ y $B$ calculados, la ecuación \ref{eqn:tubeVelocity} queda dada por:
\begin{align*}
    v(x) &=
    \frac{v_\omega}{2\iu \sin(kL)} \, e^{\iu k(L-x)}
    - \frac{v_\omega}{2\iu \sin(kL)} \, e^{\iu k(x-L)}
    \\
    &= \frac{v_\omega}{2\iu \sin(kL)} \Big[ e^{\iu k(L-x)} - e^{-\iu k(L-x)} \Big]
    \\
    &= v_\omega \, \frac{\sin(kL-kx)}{\sin(kL)}
\end{align*}

A partir de la velocidad, podemos encontrar la presión según la definición \ref{defn:particlesVelocity} integrándo la velocidad:
\begin{align*}
    p(x) &= \frac{\rho_0 \, c \, k}{\iu} \int v(x) \, \dif x
    \\
    &= \frac{\rho_0 \, c \, k}{\iu} \int v_\omega \, \frac{\sin(kL-kx)}{\sin(kL)} \, \dif x
    \\
    &= \frac{\rho_0 \, c \, k \, v_\omega}{\iu \sin(kL)} \, \int \sin(kL-kx) \, \dif x
    \\
    &= \rho_0 \, c \, v_\omega \, \frac{\cos (kL-kx)}{\iu \sin(kL)}
\end{align*}

De manera que la impedancia (Def. \ref{defn:impedance}) a lo largo del tubo está determinada por:
\begin{equation}
    Z(x) = \frac{\rho_0 \, c}{\iu \tan(kL-kx)}
\end{equation}

Por lo tanto, las frecuencias de resonancia van a ser aquellas que verifiquen que $Z(x)$ evaluado en $x_0=0$ tiende a infinito:
\begin{gather*}
    Z(x_0) \to \infty
    \iff
    \tan(kL-kx_0) \to 0
    \\
    \therefore
    k\,L = n\,\pi
\end{gather*}

O bien escribiendo $k=2\pi\, f/c$ se tiene que las frecuencias de resonancias son:
\begin{equation}
    f = \frac{n \, c}{2 \, L}
\end{equation}

O bien escribiendo $k=2\pi/\lambda$ se tiene que las longitudes de onda que pongan el tubo en resonancia son:
\begin{equation}
    \lambda = \frac{2 \, L}{n}
\end{equation}

Consideremos el caso en que el otro extremo está abierto, como se muestra en el esquema a continuación

\begin{center}
    \def\svgwidth{0.6\linewidth}
    \input{./images/tubo-cerrado-abierto.pdf_tex}
\end{center}

En el extremo abierto las partículas pueden moverse libremente con velocidad particular máxima, dándose un punto de presión mínima.
Por lo tanto, podemos asumir como hipótesis que en $x=L$ la impedancia va a ser nula.
\begin{equation*}
    \left\{
    \begin{aligned}
        p(L) &\to 0
        \\
        v(L) &\to \infty
    \end{aligned}
    \right.
    \implies
    Z \to 0
\end{equation*}

Pudiendo demostrar con un desarrollo similar al caso anterior que la impedancia está dada por:
\begin{equation}
    Z(x) = \iu \, \rho_0 \, c \tan (kL-kx)
\end{equation}

Y las frecuencias de resonancia:
\begin{equation}
    f = \frac{\left( 2n+1 \right) c}{4 \, L}
\end{equation}

Con armónicos impares de longitud de onda:
\begin{equation}
    \lambda = \frac{4 \, L}{2n + 1}
\end{equation}


\section{Modos}

Se define la frecuencia de Schroeder que determina empíricamente a partir de qué frecuencia el análisis modal de un recinto no puede aplicarse para determinar la deformada modal.

\begin{mdframed}[style=DefinitionFrame]
    \begin{defn}
    \end{defn}
    \cusTi{Frecuencia de Schroeder}
    \begin{equation*}
        f_S = 2000 \sqrt{\frac{\sub{TR}{mid}}{V}}
    \end{equation*}
\end{mdframed}

Donde $\sub{TR}{mid}$ es el tiempo de reverberación medio definido como:
\begin{equation*}
    \sub{TR}{mid} = \frac{TR(500\,\si{\hertz})+TR(1\,\si{\kilo\hertz})}{2}
\end{equation*}

Partiendo de la ecuación de Helmholtz (Def. \ref{defn:Helmholtz3D}) en 3 dimensiones para coordenadas cartesianas:
\begin{equation*}
    \grad^2 p(\Vec{x}) + k^2 \, p(\Vec{x}) = 0
\end{equation*}

Podemos disociar las variables:
\begin{equation*}
    \grad^2 \left[ p_x(x) \, p_y(y) \, p_z(z) \right] + k^2 \, p_x(x) \, p_y(y) \, p_z(z) = 0
\end{equation*}

Y al dividir por $p_x(x) \, p_y(y) \, p_z(z)$ se obtiene:
\begin{equation*}
    \frac{\grad^2 \left[ p_x(x) \, p_y(y) \, p_z(z) \right]}{p_x(x) \, p_y(y) \, p_z(z)} + k^2 = 0
\end{equation*}

Al aplicar el laplaciano, las funciones que no dependan de la variable independiente que se derive permanecen constantes:
\begin{equation}
    \frac{1}{p_x(x)} \, \frac{\partial^2 p(x)}{\partial x^2}
    + \frac{1}{p_y(y)} \, \frac{\partial^2 p(y)}{\partial y^2}
    + \frac{1}{p_z(z)} \, \frac{\partial^2 p(z)}{\partial z^2}
    + k^2 = 0
    \label{eqn:modesHelmholtz}
\end{equation}

Se define el número de onda como:
\begin{equation}
    k^2 = k_x^2 + k_y^2 + k_z^2
    \label{eqn:modesk^2}
\end{equation}

De modo que la ecuación \ref{eqn:modesHelmholtz} sea implicada por la suma de las tres ecuaciones del siguiente sistema.
\begin{gather*}
    \impliedby
    \left\{
    \begin{aligned}
        \frac{1}{p_x(x)} \, \frac{\partial^2 p(x)}{\partial x^2} + k_x^2 = 0
        \\
        \frac{1}{p_y(y)} \, \frac{\partial^2 p(y)}{\partial y^2} + k_y^2 = 0
        \\
        \frac{1}{p_z(z)} \, \frac{\partial^2 p(z)}{\partial z^2} + k_z^2 = 0
    \end{aligned}
    \right.
    \\
    \iff
    \left\{
    \begin{aligned}
        \frac{\partial^2 p(x)}{\partial x^2} + k_x^2 \, p_x(x) = 0
        \\
        \frac{\partial^2 p(y)}{\partial y^2} + k_y^2 \, p_y(y) = 0
        \\
        \frac{\partial^2 p(z)}{\partial z^2} + k_z^2 \, p_z(z) = 0
    \end{aligned}
    \right.
\end{gather*}

Siendo cada solución $p_x(x)$, $p_y(x)$ y $p_z(x)$ del tipo $p_x(x) = C_x \cos (k_x x + \varphi)$ luego la velocidad de partículas está dada análogamente según la definición \ref{defn:particlesVelocity} por:
\begin{equation*}
    v_x(x) = \frac{\iu}{\rho_0 \, c \, k_x} \, \frac{\partial p_x(x)}{\partial x} = - \frac{\iu \, C_x}{\rho_0 \, c} \sin (k_x x + \varphi)
\end{equation*}

Por hipótesis de contorno, se tiene que cumplir para $x_0=0$ que $v(x_0) = 0$ y $v(L_x) = 0$ simultáneamente.
\begin{gather*}
    v(x_0) = - \frac{\iu \, C_x}{\rho_0 \, c} \sin (\varphi) = 0
    \\
    \implies \varphi = 0
    \\
    \therefore \, v(L_x) = - \frac{\iu \, C_x}{\rho_0 \, c} \sin (k_x L_x) = 0
    \\
    \implies k_x L_x = n_x \pi \implies k_x = \frac{n_x \pi}{L_x}
\end{gather*}

Reemplazando $k_x$, y luego con el mismo razonamiento $k_y$ y $k_z$, en la ecuación \ref{eqn:modesk^2} queda definida la frecuencia de resonancia propia del recinto.
\begin{gather*}
    k^2 = \left( \frac{\omega}{c} \right)^2 =
    \\
    \left( \frac{2\pi\,f_0}{c} \right)^2
    = \left( \frac{n_x \, \pi}{L_x} \right)^2 + \left( \frac{n_y \, \pi}{L_y} \right)^2 + \left( \frac{n_z \, \pi}{L_z} \right)^2
\end{gather*}

\begin{mdframed}[style=DefinitionFrame]
    \begin{defn}
    \end{defn}
    \cusTi{Frecuencias modales}
    \cusTe{Frecuencias de resonancia que excitan los modos propios de un recinto rectangular.}
    \begin{equation*}
        f_0 = \frac{c}{2} \sqrt{\left(\frac{n_x}{L_x}\right)^2 + \left(\frac{n_y}{L_y}\right)^2 + \left(\frac{n_z}{L_z}\right)^2}
    \end{equation*}
\end{mdframed}


\section{Criterios empíricos}


\subsection{Criterio de Bolt}

Richard H. Bolt propone en 1946 que los recintos de óptima escucha serán los que tengan una distribución frecuencial homogénea entre modos.
El criterio de Bolt se verifica para salas con dimensiones proporcionales a las siguientes siendo $L_z$ unitario:

\begin{equation*}
    \left\{
    \begin{aligned}
        & \frac{3}{2} \left( L_x-1 \right) < L_y -1 < 3 \left( L_x-1 \right)
        \\
        & 2 < L_x+L_y < 4
    \end{aligned}
    \right.
\end{equation*}


\subsection{Criterio de Bonello}

Oscar J. Bonello determina en 1981 un criterio para la mejor distribución de los modos normales de una sala a partir de la \emph{densidad modal} o cantidad de modos por debajo de cierta frecuencia.
\begin{equation*}
    N(f) = \frac{4\pi\,V}{3} \left(\frac{f}{c}\right)^3 + \frac{\pi \, S}{4} \left(\frac{f}{c}\right)^2 + \frac{L}{8} \left(\frac{f}{c}\right)
\end{equation*}

De manera que una sala será libre de coloración si:
\begin{itemize}
    \item La cantidad de modos en función de la frecuencia es monótonamente creciente.
    \item No tiene modos dobles para bandas con menos de 5 modos
\end{itemize} 


\subsection{Criterio de Cox}

Trevor J. Cox y Peter D'Antonio establecen en 2001 un método para determinar las dimensiones óptimas de una sala para la escucha crítica.
Estudiando la función de transferencia entre sus esquinas, adecuaron las dimensiones para que esta sea lo más plana posible.