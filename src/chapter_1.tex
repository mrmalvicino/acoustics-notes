\chapter{Vibraciones}

\begin{mdframed}[style=DefinitionFrame]
    \begin{defn}
    \end{defn}
    \cusTi{Vibración}
    \cusTe{Una vibración es el movimiento de una partícula al rededor de un punto de equilibrio.}
\end{mdframed}

En varias ocasiones, el comportamiento de las vibraciones se estudia a partir de oscilaciones en función del tiempo, independientemente de la propagación del disturbio en el espacio que se pueda llegar a generar.
Ahora bien, a menos que se estudie el movimiento armónico de una partícula aislada en el vacío, toda vibración se da en un medio continuo y por tanto va a devenir en sonido, ya que se va a propagar en el espacio por estar en interacción con su entorno.

En la jerga, se suelen asociar las vibraciones a oscilaciones de baja frecuencia dadas en un medio sólido.
Pero estrictamente sería correcto decir, por ejemplo, que una partícula de aire está vibrando al ser perturbada por un sonido.
Ergo, las vibraciones son sonido y viceversa.
La única diferencia es el enfoque de estudio que se le dé al fenómeno.

Si bien resulta curioso que ciertos profesores no conozcan este tecnicismo, doy fe que los lectores de este texto sabrán apreciarlo.

\section{Resortes equivalentes}

Las vibraciones son estudiadas mediante sistemas masa-resorte.
Por más que un sistema esté compuesto por varios resortes reales, se lo puede reducir a un sistema más simple compuesto por un único resorte ideal equivalente.

Los resortes en un sistema mecánico se comportan de manera similar a como se comportarían los capacitores en un sistema eléctrico, ya que almacenan energía potencial.

En el esquema siguiente se muestran dos sistemas compuestos por una masa colgando de dos resortes con un extremo fijo en el techo.
Los resortes en el sistema de la izquierda están dispuestos en paralelo y los del sistema de la derecha en serie.

\begin{center}
    \def\svgwidth{0.8\linewidth}
    \input{./images/k-serie-paralelo.pdf_tex}
\end{center}

Se supone como hipótesis que el peso de la masa está perfectamente equilibrado y que esta solo tiene un grado de libertad para desplazarse.

\begin{mdframed}[style=DefinitionFrame]
    \begin{defn}
    \end{defn}
    \cusTi{Constante de rigidez equivalente}
    \cusTe{Resortes en paralelo:}
    \begin{equation*}
        k = \sum_{\ith=1}^\nth k_\ith
    \end{equation*}
    Resortes en serie:
    \begin{equation*}
        k = \frac{1}{\sum\limits_{\ith=1}^\nth \frac{1}{k_\ith}}
    \end{equation*}
\end{mdframed}


\section{Deflexión estática}

La deflexión estática es el estiramiento de un resorte que tiene en un extremo una masa adosada debido al peso de la misma por encontrarse el sistema dispuesto de manera vertical.
Para un sistema con origen en la longitud natural del resorte, la deflexión estática corresponde a su punto de equilibrio de manera que toda oscilación que eventualmente se produzca va a darse en torno a este.


\subsection*{Deflexión estática en resortes de expansión}

En el esquema a continuación se muestran tres instancias en las que se tiene un resorte de expansión de longitud natural $l_0$ con un extremo fijo.
El otro extremo está libre y eventualmente puede moverse.

\begin{center}
    \def\svgwidth{0.8\linewidth}
    \input{./images/deflec-estat-techo.pdf_tex}
\end{center}

En principio, el resorte está descargado.
Está en reposo, y la posición del extremo libre coincide con la longitud natural.
Al no tener peso, la deflexión estática es nula.
Luego se lo carga con una masa $m_0$, estirándose hasta que se detiene y permanece en equilibrio.
Debido al peso de esta masa, se produce una deflexión estática $\Delta x_0=x_0-l_0$.
Finalmente, se agrega una masa $\delta m_1$ y el resorte queda cargado con una masa $m_1=m_0+\delta m_1$ que hace que se estire aún más y genere una deflexión estática $\Delta x_1=x_1-l_0$ mayor a la anterior.

Este proceso podría repetirse iteradamente aumentando el valor de la masa, obteniendo así una deflexión estática cada vez mayor.
Al agregar un incremento de masa conocida ($\delta m_\ith$), la masa total ($m_\ith$) está dada por la masa del sistema más el agregado.
De manera general, decimos que para cierta masa $m_\ith=m_0+\delta m_\ith$ el resorte presenta una deflexión estática $\Delta x_\ith=x_\ith-l_0$ que se calcula por leyes de Newton como sigue.
\begin{gather*}
    \sum F = 0 = m_\ith \, g - k \, \Delta x_\ith
    \\
    \frac{\Delta x_\ith}{m_\ith} = \frac{g}{k}
\end{gather*}

Esta relación es útil para calcular la deflexión estática de un sistema cuya masa ($m_0$) es desconocida.

Aplicándola para la masa propia del sistema $m_0$ y luego para cierta masa $m_1=m_0 + \delta m_1$ podemos reescribir las deflexiones correspondientes de la siguiente manera.
\begin{equation*}
    \left\{
    \begin{aligned}
        \frac{g}{k} &= \frac{\Delta x_0}{m_0}
        \\
        \frac{g}{k} &= \frac{\Delta x_1}{m_1} = \frac{\Delta x_0 + x_1 - x_0}{m_0 + \delta m_1}
    \end{aligned}
    \right.
\end{equation*}

O bien de manera general:
\begin{equation*}
    \left\{
    \begin{aligned}
        \frac{g}{k} &= \frac{\Delta x_0}{m_0}
        \\
        \frac{g}{k} &= \frac{\Delta x_0 + (x_\ith - x_0)}{m_0 + \delta m_\ith}
    \end{aligned}
    \right.
\end{equation*}

Se despeja de la segunda ecuación del sistema:
\begin{align*}
    \frac{g}{k} \left( m_0 + \delta m_\ith \right) &= \Delta x_0 + (x_\ith - x_0)
    \\
    \underbrace{\frac{g}{k} \, m_0}_{\Delta x_0} + \frac{g}{k} \, \delta m_\ith &= \Delta x_0 + (x_\ith - x_0)
\end{align*}

Y restando $\Delta x_0$ de ambos miembros se llega a la conclusión de que la deflexión estática es independiente de la longitud natural.
\begin{equation*}
    \frac{g}{k} = \frac{x_\ith-x_0}{\delta m_\ith} = \frac{\Delta x_\ith}{m_\ith}
\end{equation*}


\subsection*{Deflexión estática en resortes de compresión}

Una situación análoga puede analizarse para un resorte de compresión, obteniendo el mismo resultado con signo contrario por no definir la fuerza elástica por convención.

\begin{center}
    \def\svgwidth{0.8\linewidth}
    \input{./images/deflec-estat-piso.pdf_tex}
\end{center}

Una masa $m_\ith = m_0+\delta m_\ith$ se corresponde con una deflexión estática $\Delta x_\ith = l_0 - x_\ith$ según:
\begin{gather*}
    \sum F = 0 = -m_\ith \, g + k \, \Delta x_\ith
    \\
    \frac{\Delta x_\ith}{m_\ith} = \frac{g}{k}
\end{gather*}

Particularmente, para la masa propia del sistema $m_0$ se tiene:
\begin{equation*}
    \Delta x_0 = \frac{m_0 \, g}{k}
\end{equation*}

Y de manera general, habiendo incrementado la masa, se tiene:
\begin{equation*}
    \frac{g}{k} = \frac{\Delta x_\ith}{m_\ith} = \frac{\Delta x_0 + (x_0-x_\ith)}{m_0 + \delta m_\ith}
\end{equation*}

Reemplazando $\Delta x_0$ en la ecuación anterior y despejando queda:
\begin{equation*}
    \frac{g}{k} = \frac{x_0 - x_\ith}{\delta m_\ith} = \frac{\Delta x_\ith}{m_\ith}
\end{equation*}


\section{Oscilación amortiguada}

Se tiene un sistema masa-resorte en un medio viscoso y se pretende estudiar el coeficiente de amortiguamiento ($R_a$) del sistema modelándolo como un movimiento armónico amortiguado.

La ecuación diferencial que rige el movimiento sale de hacer la sumatoria de fuerzas sobre la masa según las leyes de Newton:
\begin{align*}
    \sum F = m \, a &= -\sub{F}{ela} - \sub{F}{roz}
    \\
    m \, \ddot{x} &= -k \, x - R_a \, \dot{x}
    \\
    \ddot{x} + 2 \, \xi \, \omega_0 \, \dot{x} + \omega_0^2 \, x &= 0
\end{align*}

Donde $\omega_0=\sqrt{\frac{k}{m}}$ es la frecuencia angular natural y $\xi=\frac{R_a}{2 \, m \, \omega_0}$ es el índice de amortiguación.

La familia de soluciones generales es la siguiente. \cite{1} (Pág.~85)

\begin{equation}
    x(t)=A\,e^{-\tfrac{R_a}{2m}t} \, \cos (\omega_d\,t+\varphi)
    \label{eqn:MAA}
\end{equation}

Donde $\omega_d=\sqrt{\omega_0^2 - \left(\tfrac{R_a}{2m}\right)^2 }$ es la pseudo pulsación del sistema.

Resulta evidente que el valor de la constante de amortiguamiento ($R_a$) va a determinar la envolvente exponencial que modela la atenuación en la forma de onda de la oscilación.
Si el sistema tiene mayor amortiguamiento, más abrupto va a ser el decaimiento de la oscilación.
Por este motivo, es importante analizar la relación entre máximos o mínimos sucesivos.

Una característica de las funciones exponenciales es que si tomamos intervalos de tiempo constantes observamos un decaimiento proporcional al valor que toma la función en el intervalo anterior.
El factor de proporción es constante.
Lo notamos como $\lambda$ y determina qué porcentaje de amplitud tiene la oscilación al cabo de $n$ ciclos siguientes.
Se calcula mediante:
\begin{equation*}
    \dfrac{x(t_0+nT)}{x(t_0)}=\lambda^n
\end{equation*}

Por otro lado, podemos plantear por definición (Ec. \ref{eqn:MAA}) la relación de amplitud entre dos máximos separados por $n$ ciclos:
\begin{equation*}
    \dfrac{x(t_0+nT)}{x(t_0)} = \dfrac
    {A\,e^{-\tfrac{R_a}{2m}(t_0+nT)} \, \sin \left[\omega_d (t_0+nT)+\varphi \right]}
    {A\,e^{-\tfrac{R_a}{2m} t_0} \, \sin \left(\omega_d t_0+\varphi \right)}
\end{equation*}

La amplitud \emph{máxima} ($A$) se simplifica para ambos puntos.
Las funciones periódicas valen lo mismo evaluadas en un punto que evaluadas en un punto más $n$ períodos, por lo que se simplifican.
Restando los exponentes de las exponenciales, queda:
\begin{equation*}
    \dfrac{x(t_0+nT)}{x(t_0)} = e^{-\tfrac{R_a}{2m} n T}
\end{equation*}

Se iguala la relación anterior dada por definición, con la relación dada por anlálisis de máximos:
\begin{align*}
    e^{-\tfrac{R_a}{2m} nT} &= \lambda^n
    \\
    -\dfrac{R_a}{2m} \, n \, T &= \ln (\lambda^n)
\end{align*}

Observar que el período denotado anteriormente como $T$ de manera general para cualquier función exponencial, es aquel asociado a la pseudo pulsación $\omega_d$ para el decaimiento de la amplitud:
\begin{equation*}
    T = T_d \approx T_0
\end{equation*}

El coeficiente de amortiguamiento ($R_a$) puede ser calculado mediante dos métodos.

Por un lado, se puede despejar $R_a$ mediante la aproximación:
\begin{equation*}
    \omega_d \approx \omega_0 \implies T_d \approx T_0
\end{equation*}

Sabiendo que la pseudo pulsación $\omega_d$ de la oscilación es apenas menor que la pulsación natural $\omega_0$ que tendría el sistema si no fuese amortiguado.
\begin{align*}
    -\dfrac{R_a \, n}{2m} \cdot \dfrac{2 \pi}{\omega_0} &\approx \ln (\lambda^n)
    \\
    R_a &\approx -\dfrac{m \, \omega_0 \ln (\lambda)}{\pi}
\end{align*}

Por otro lado, definiendo el pseudo período:
\begin{equation*}
    T_d=\frac{2\pi}{\omega_d}
\end{equation*}

Donde:
\begin{equation*}
    \omega_d=\sqrt{\omega_0^2-\left(\frac{R_a}{2m}\right)^2}
\end{equation*}

Y luego despejando:
\begin{align*}
    -\dfrac{R_a}{2m} \, n \, \dfrac{2 \pi}{\sqrt{\omega_0^2-\left(\frac{R_a}{2m}\right)^2}} &= \ln (\lambda^n)
    \\
    \left( \frac{2 \pi \, n \, R_a}{2m} \right)^2 \frac{1}{\omega_0^2-\left(\frac{R_a}{2m}\right)^2} &= \ln^2 (\lambda^n)
    \\
    \left( \frac{\pi \, n \, R_a}{m} \right)^2 &= \Big[ n \ln(\lambda) \Big]^2 \left[ \omega_0^2 - \left( \frac{R_a}{2m} \right)^2 \right]
    \\
    \left( \frac{\pi \, R_a}{m} \right)^2 &= \omega_0^2 \, \ln^2 (\lambda) - \left( \frac{R_a}{2m} \right)^2 \ln^2 (\lambda)
    \\
    \left[ \frac{\ln (\lambda) \, R_a}{2m} \right]^2 + \left( \frac{\pi \, R_a}{m} \right)^2 &= \Big[ \omega_0 \, \ln (\lambda) \Big]^2
    \\
    \left[ \frac{\ln (\lambda) \, R_a}{2} \right]^2 + (\pi \, R_a)^2 &= \Big[ m \, \omega_0 \, \ln (\lambda) \Big]^2
    \\
    R_a^2 &= \frac{\Big[ m \, \omega_0 \, \ln (\lambda) \Big]^2}{\dfrac{\ln^2(\lambda)}{4}+\pi^2}
    \\
    R_a &= \frac{m \, \omega_0 \norm{\ln (\lambda)}}{\sqrt{\dfrac{\ln^2(\lambda)}{4}+\pi^2}}
\end{align*}


\section{Oscilación forzada}
\label{sec:oscilacionForzada}

El movimiento de un sistema masa-resorte al que se le aplica una fuerza periódica se modela como un movimiento armónico amortiguado forzado.
En la figura a continuación se puede ver un esquema del sistema, en el que se tiene una masa unida al extremo de un resorte ideal con el otro extremo fijo.

\begin{center}
    \def\svgwidth{0.5\linewidth}
    \input{./images/osc-maaf-diag.pdf_tex}
\end{center}

Haciendo la sumatoria de fuerzas sobre la masa según las leyes de Newton, se obtiene la ecuación diferencial que rige el movimiento de la misma.
\begin{align*}
    \sum F_x = m a &= \sub{F}{ext} - \sub{F}{ela} - \sub{F}{fri}
    \\
    m \, \ddot{x} &= \sub{F}{ext} - k \, x - R_a \, \dot{x}
    \\
    \ddot{x} + 2 \, \xi \, \omega_0 \, \dot{x} + \omega_0^2 \, x &= \frac{\sub{F}{ext}}{m}
\end{align*}

Donde $\omega_0=\sqrt{\frac{k}{m}}$ es la frecuencia angular natural y $\xi=\frac{R_a}{2 \, m \, \omega_0}$ es el índice de amortiguación.

Cuya solución general está dada por la siguiente función del tiempo. \cite{1} (Pág.~92)
\begin{equation*}
    x(t) = A e^{-\tfrac{\gamma}{2}t} \cos{(\omega_d t + \varphi)} + |X| \cos (\omega t + \theta)
\end{equation*}

Donde:
\begin{align*}
    \norm{X} &= \frac{F_0}{m \sqrt{(\omega_0^2-\omega^2)^2+(2 \, \xi \, \omega_0 \, \omega)^2}}
    \\
    \theta &= \arctan \left( \frac{-\frac{R_a \, \omega}{m}}{\omega_0^2 - \omega^2} \right)
\end{align*}

O bien, sacando factor común $\omega_0$ y definiendo la relación $r = \frac{\omega}{\omega_0}$ se tiene:
\begin{align*}
    \norm{X} &= \frac{F_0}{k \sqrt{(1-r^2)^2+(2 \xi r)^2}}
    \\
    \theta &= \arctan \left( - \frac{2 \, \xi \, r}{1-r^2} \right)
\end{align*}

Pudiendo definir $D=\frac{1}{\sqrt{(1-r^2)^2+(2 \xi r)^2}}$ como la amplitud dinámica tal que $\norm{X}=\frac{F_0 \, D}{k}$.

Nótese que $\theta$ es el defasaje entre la posición de la masa y la fuerza externa.
No debe ser confundida con $\varphi$ que es la fase de la oscilación amortiguada que depende de las condiciones iniciales del sistema.
En la siguiente figura se dan, para distintos valores de $\xi$, gráficos del defasaje $\theta(r)$ que depende de la frecuencia de la fuerza externa.

\begin{center}
    \def\svgwidth{\linewidth}
    \input{./images/osc-maaf-fase.pdf_tex}
\end{center}

La frecuencia de la fuerza externa es positiva lo cual implica que $r>0$.
Por lo tanto, para interpretar el gráfico de fase, solo se debe considerar la mitad derecha del mismo.


\section{Transmisibilidad}

A partir del modelo de oscilación forzada (Sec. \ref{sec:oscilacionForzada}) resulta de interés estudiar cómo se transmite la fuerza externa en el otro extremo del resorte considerando que ninguno de los extremos están fijos.

Se tiene un sistema masa-resorte en un medio viscoso al que se le impone una fuerza externa de magnitud periódica en cualquiera de los extremos.

\begin{center}
    \def\svgwidth{0.8\linewidth}
    \input{./images/vib-transmisibilidad-diag.pdf_tex}
\end{center}

Como hipótesis se supone que todos los puntos que componen la masa $m$ se mueven con un solo grado de libertad y con la misma aceleración.
Estas condiciones se dan análogamente para los puntos de la base donde está el sistema apoyado, o el techo en caso que esté colgado.
Ergo, todos los puntos del bloque $A1$ se modelan mediante $x(t)$ y todos los puntos del bloque $A2$ mediante $y(t)$.

Se pretende determinar cómo los dos bloques $A1$ y $A2$ interactúan entre si mediante un resorte real eventualmente actuando en conjunto con un amortiguador real, que puede o no existir.
El modelo usa un resorte ideal y un amortiguador ideal ya que, por más que no se tenga un amortiguador real, el resorte real y el rozamiento con el medio tienen un comportamiento disipativo similar al de un amortiguador.

Para estudiar el efecto de amortiguación que tenga el sistema se pretende analizar cómo la fuerza, aplicada sobre uno de los bloques, influye en el movimiento del otro bloque.
Para esto se estudia la relación entre el desplazamiento $x(t)$ e $y(t)$ de los bloques conocida como \emph{transmisibilidad}.

\begin{mdframed}[style=DefinitionFrame]
    \begin{defn}
    \end{defn}
    \cusTi{Transmisibilidad}
    \cusTe{Relación entre amplitud de la vibración y amplitud transmitida.}
    \begin{equation*}
        T_F = \frac{\norm{X}}{\norm{Y}}
    \end{equation*}
\end{mdframed}

Se define, además, el nivel de transmisibilidad en escala logarítmica como sigue.

\begin{mdframed}[style=DefinitionFrame]
    \begin{defn}
    \end{defn}
    \cusTi{Nivel de transmisibilidad}
    \cusTe{Valor de transmisibilidad en decibeles.}
    \begin{equation*}
        L_T=10 \log \left( T_F \right)
    \end{equation*}
\end{mdframed}

Si uno de los extremos estuviese fijo, la compresión del resorte estaría dada solo por la posición del otro extremo.
Pero como ambos extremos tienen posiciones variables, la compresión neta depende de la diferencia entre $x(t)$ e $y(t)$ ya que una compresión en uno de los extremos podría compensarse por una extensión en el otro extremo obteniendo un efecto nulo en la fuerza elástica.
Para sortear esta cuestión, hay que considerar en la ecuación diferencial que rige el movimiento la diferencia de posición y velocidad entre ambos extremos.
Dado que se quiere analizar la aceleración en $x(t)$, se considera que el otro extremo está en equilibrio cuando la fuerza externa es nula.

\begin{align}
    m\,\Ddot{x} + R_a \left( \Dot{x}-\Dot{y} \right) + k \left( x-y \right) &= 0
    \notag
    \\
    \Ddot{x} + \frac{R_a}{m} \left( \Dot{x}-\Dot{y} \right) + \frac{k}{m} \left( x-y \right) &= 0
    \notag
    \\
    \Ddot{x}+2\xi\,\omega_0\left(\Dot{x}-\Dot{y}\right) + \omega_0^2 \left( x-y \right) &= 0
    \label{eqn:MAAF}
\end{align}

Donde $\omega_0=\sqrt{\frac{k}{m}}$ es la frecuencia natural del sistema y $\xi=\tfrac{R_a}{2\omega_0\,m}$ su índice de amortiguamiento.

Las soluciones particulares de la ecuación \ref{eqn:MAAF} son:
\begin{equation*}
    \left\{
    \begin{aligned}
        x(t) &= \norm{X} e^{\iu (\omega \, t + \theta)}
        \\
        y(t) &= \norm{Y} e^{\iu \, \omega \, t}
    \end{aligned}
    \right.
\end{equation*}

Y sus derivadas:
\begin{equation*}
    \left\{
    \begin{aligned}
        \Dot{x}(t) &= \iu \omega \norm{X} e^{\iu (\omega \, t + \theta)}
        \\
        \Dot{y}(t) &= \iu \omega \norm{Y} e^{\iu \, \omega \, t}
    \end{aligned}
    \right.
\end{equation*}

Mientras que la segunda derivada de $x(t)$ es:
\begin{equation*}
    \Ddot{x}(t) = -\omega^2 \norm{X} e^{\iu (\omega \, t + \theta)}
\end{equation*}

Reemplazando las soluciones y sus derivadas en la ecuación \ref{eqn:MAAF} se tiene una ecuación que relaciona $\norm{X}$ y $\norm{Y}$ como sigue:
\begin{multline*}
    -\omega^2 \norm{X} e^{\iu (\omega t + \theta)} +
    \\
    + 2 \, \xi \, \omega_0 \left[ \iu \omega \norm{X} e^{\iu (\omega t + \theta)} - \iu \omega \norm{Y} e^{\iu \omega t} \right] +
    \\
    + \omega_0^2 \left[ \norm{X} e^{\iu (\omega \, t + \theta)} - \norm{Y} e^{\iu \, \omega \, t} \right] = 0
\end{multline*}

\begin{multline*}
    -\omega^2 \norm{X} e^{\iu (\omega t + \theta)} +
    \\
    + 2 \, \xi \, \omega_0 \, \iu \, \omega \norm{X} e^{\iu (\omega t + \theta)} - 2 \, \xi \, \omega_0 \, \iu \, \omega \norm{Y} e^{\iu \omega t} +
    \\
    + \omega_0^2 \norm{X} e^{\iu (\omega \, t + \theta)} - \omega_0^2 \norm{Y} e^{\iu \, \omega \, t} = 0
\end{multline*}

\begin{multline*}
    \norm{X} e^{\iu (\omega \, t + \theta)} \left[ -\omega^2 + 2 \, \xi \, \omega_0 \, \iu \, \omega + \omega_0^2  \right] =
    \\
    = \norm{Y} e^{\iu \, \omega \, t} \left[ 2 \, \xi \, \omega_0 \, \iu \, \omega + \omega_0^2 \right]
\end{multline*}

\begin{align*}
    \frac{\norm{X}}{\norm{Y}} \, e^{\iu\theta}
    &= \frac{\omega_0^2 + \iu \, 2 \, \xi \, \omega_0 \, \omega}{\omega_0^2 - \omega^2 + \iu \, 2 \, \xi \, \omega_0 \, \omega}
    \\
    &= \frac{\omega_0^2}{\omega_0^2} \cdot \frac{1+ \iu \, 2 \xi \frac{\omega}{\omega_0}}{1-\frac{\omega^2}{\omega_0^2} + \iu \, 2 \, \xi \frac{\omega}{\omega_0}}
\end{align*}

\begin{mdframed}[style=DefinitionFrame]
    \begin{defn}
    \end{defn}
    \cusTi{Relación de frecuencias}
    \cusTe{Relación entre la frecuencia de excitación y la frecuencia de resonancia.}
    \begin{equation*}
        r=\frac{\omega}{\omega_0}
    \end{equation*}
\end{mdframed}

Tomando módulo en ambos miembros se tiene:
\begin{align*}
    \frac{\norm{X}}{\norm{Y}} \, e^{\iu\theta}
    &= \frac{1+ \iu \, 2 \xi \, r}{1-r^2 + \iu \, 2 \, \xi \, r}
    \\
    \norm{\frac{\norm{X}}{\norm{Y}} \, e^{\iu\theta}}
    &= \norm{\frac{1+ \iu \, 2 \, \xi \, r}{1-r^2 + \iu \, 2 \, \xi \, r}}
    = \frac{\norm{1 + \iu \, 2 \, \xi \, r}}{\norm{1-r^2 + \iu \, 2 \, \xi \, r}}
    \\
    \frac{\norm{X}}{\norm{Y}} \, \underbrace{\norm{e^{\iu\theta}}}_1
    &= \frac{\sqrt{1 + (2 \, \xi \, r)^2}}{\sqrt{(1-r^2)^2 + (2 \, \xi \, r)^2}}
\end{align*}

\begin{mdframed}[style=PropertyFrame]
    \begin{prop}
    \end{prop}
    \begin{equation*}
        \frac{\norm{X}}{\norm{Y}} = \sqrt{\frac{1 + (2 \, \xi \, r)^2}{(1-r^2)^2 + (2 \, \xi \, r)^2}}
    \end{equation*}
\end{mdframed}

Observar que $\norm{X}$ e $\norm{Y}$ son los módulos de las funciones de posición $x(t)$ e $y(t)$ cuyas partes reales determinan el desplazamiento de cada extremo $A1$ y $A2$ y son soluciones particulares de la ecuación \ref{eqn:MAAF}.

Al graficar el nivel de transmisibilidad, se obtiene la siguiente familia de curvas que dependen del parámetro $0<\xi<1$ que caracteriza cada sistema.

\begin{center}
    \def\svgwidth{\linewidth}
    \input{./images/vib-transmisibilidad-nivel.pdf_tex}
\end{center}

Se puede observar en la figura anterior que para valores de $r>\sqrt{2}$ el sistema va a estar amortiguando la fuerza externa, ya que $\norm{Y}>\norm{X}$, lo cual implica que el desplazamiento en el extremo donde se aplica la fuerza es mayor al extremo sobre el otro.

El sistema va a estar en resonancia cuando la frecuencia de la fuerza impuesta sea igual a la frecuencia natural.
Cuando $r=\frac{\omega}{\omega_0}=1$ el sistema es exitado a su frecuencia de resonancia y la oscilación se amplifica.
Pero la frecuencia de resonancia es inversamente proporcional a el índice de amortiguamiento debido a que $\xi$ depende de $\omega_0$.
Por esto, se observa en el gráfico de $L_T(r)$ que el punto máximo es distinto para las diferentes curvas.

La fuerza externa es $\sub{F}{ext}=F_0 \cos (\omega t)$ por ser su magnitud periódica.
Existe una relación entre $\sub{F}{ext}$ y el desplazamiento $\norm{X}$ que la fuerza genera.
\begin{align*}
    F_0 \cos (\omega t) &= m \, \ddot{x}
    \\
    F_0 \iint \cos (\omega t) \, \dif t^2 &= m \iint \ddot{x} \, \dif t^2
    \\
    -\frac{F_0 \cos (\omega t)}{\omega^2} &= m \, x(t)
    \\
    -\frac{F_0 \cos (\omega t)}{\omega^2} &= m \, \norm{X} \cos (\omega t)
\end{align*}

\begin{mdframed}[style=PropertyFrame]
    \begin{prop}
    \end{prop}
    \begin{equation*}
        F_0 = m \, \norm{X} \, \omega^2
    \end{equation*}
\end{mdframed}

Analizando la fuerza transmitida $\sub{F}{tra}=F_1 \cos(\omega t)$ y el desplazamiento $\norm{Y}$ en el otro extremo, se obtiene $F_1 = m \, \norm{Y} \, \omega^2$ para el máximo de fuerza transmitida de modo que:

\begin{mdframed}[style=PropertyFrame]
    \begin{prop}
    \end{prop}
    \begin{equation*}
        \frac{\norm{X}}{\norm{Y}}=\frac{F_0}{F_1}
    \end{equation*}
\end{mdframed}