\chapter{El fenómeno sonoro}

\begin{mdframed}[style=DefinitionFrame]
    \begin{defn}
        \label{defn:sound}
    \end{defn}
    \cusTi{Sonido}
    \cusTe{El sonido es una variación de presión o desplazamiento de partículas que se propaga en un medio elástico a causa de una perturbación que sea detectable por un instrumento tecnológico o mediante el órgano receptor en un ser vivo.}
\end{mdframed}

\section{Forma de onda}

Una onda es la caracterización de la propagación del sonido en espacio y tiempo.
La representación de una onda para un punto fijo del espacio conforme transcurre el tiempo se conoce como \emph{forma de onda}.

En el siguiente gráfico se tiene un ejemplo para la forma de onda de la presión acústica.
Se observa que esta toma distintos valores en función del tiempo en torno a $p_0$, que es el promedio.

\begin{center}
    \def\svgwidth{0.8\linewidth}
    \input{./images/pressure-waveform.pdf_tex}
\end{center}

Siendo $p_0$ la presión atmosférica estándar:
\begin{equation*}
    p_0 = 1 \,\si{\atmosphere} = 1013 \,\si{\milli\bar} \approxeq 10^5 \,\si{\pascal} = 10^5 \,\si{\newton\per\metre^2}
\end{equation*}


\section{Hipótesis de la acústica lineal}
\label{sec:linAcousticHyp}

La acústica lineal estudia los fenómenos sonoros que presentan un nivel de presión sonora relativamente bajo, menor a $160\,\si{\deci\bel}$ aproximadamente.
Las perturbaciones que generan una variación de presión no lineal no pueden ser modeladas matemáticamente de manera predecible por la ecuación de onda.

Una perturbación comprime el volumen aumentando su presión y luego expande el volumen disminuyendo la presión.
Esta transformación se da de manera periódica conforme a la frecuencia de excitación, de modo que no hay tiempo suficiente para que se genere transferencia de calor entre dos momentos de máxima presión.

Suponemos entonces que:

\begin{itemize}
    \item El desplazamiento de las partículas al rededor de su punto de equilibrio es chico.

    \item Las variaciones de presión y densidad son bajas en relación a las magnitudes en ausencia de una perturbación.
    
    \item El medio es homogéneo.
    Ante la ausencia de un disturbio, la densidad y la presión valen $\rho_0$ y $p_0$ respectivamente para todos los puntos.
    
    \item El rozamiento del medio es despreciable.
    No hay pérdidas de energía por disipación.
    
    \item Las distancias son menores a $100\,\si{\metre}$.
    Se desprecia la pérdida de energía por divergencia.
    
    \item La transformación efectuada por una onda de sonido no es isoterma sino adiabática, verificando que $p\,V^\gamma$ es constante.
\end{itemize}

Pudiendo desarrollar:

\begin{align*}
    \Delta \left( p\,V^\gamma \right) &= 0
    \\
    p_1\,V_1^\gamma &= p_0\,V_0^\gamma
    \\
    p_1 &= p_0 \left(\frac{V_0}{V_1} \right)^\gamma
    \\
    p_1 &= p_0 \left( \frac{\tfrac{m}{\rho_0}}{\tfrac{m}{\rho_1}} \right)^\gamma
    \\
    p_1 &= p_0 \left( \frac{\rho_1}{\rho_0} \right)^\gamma
\end{align*}

A continuación se grafica la relación obtenida para $p_1$ en función de $\rho_1$ con $\gamma=1.4$ y se muestra la recta tangente a fin que pasa por $[\rho_0;p_0]$.

\begin{center}
    \def\svgwidth{0.8\linewidth}
    \input{./images/pressure-density.pdf_tex}
\end{center}

Dado que existe la derivada de la función $p_1(\rho_1)$, vemos que la recta tangente que pasa por $p_1(\rho_0)$ es una buena aproximación lineal.
\begin{equation*}
    \frac{\dif}{\dif \rho_1} p_1(\rho_1) = \frac{p_0 \, \gamma}{\rho_0^\gamma} \, \rho_1^{\gamma-1}
\end{equation*}

Evaluando la derivada en $\rho_1=\rho_0$ se tiene:
\begin{equation*}
    \frac{\dif}{\dif \rho_1} p_1(\rho_0) = \frac{p_0 \, \gamma}{\rho_0^\gamma} \, \rho_0^{\gamma-1} = \frac{p_0 \, \gamma}{\rho_0}
\end{equation*}

Se plantea la forma diferencial $\dif p_1$ que es la aproximación lineal de $p_1$:
\begin{align*}
    \dif p_1 = \frac{p_0 \, \gamma}{\rho_0} \, \dif \rho_1 = \frac{p_0 \, \gamma}{\rho_0} \, \Delta \rho_1 & \approx \Delta p_1
    \\
    \frac{p_0 \, \gamma}{\rho_0} \left(\rho_1-\rho_0\right) & \approx \left(p_1-p_0\right)
\end{align*}

Se define la presión acústica $p=p_1-p_0$ como la diferencia entre la presión total y la presión del fluido sin perturbar.
Análogamente se define $\rho=\rho_1-\rho_0$ para la densidad.
Quedando finalmente:

\begin{mdframed}[style=PropertyFrame]
    \begin{prop}
        \label{prop:linearAcoustics}
    \end{prop}
    \cusTi{Hipótesis de la acústica lineal}
    \begin{equation*}
        p \approx c^2\,\rho
    \end{equation*}
\end{mdframed}


Donde $c^2=\tfrac{p_0 \, \gamma}{\rho_0}$ es la velocidad de propagación del sonido, verificable en el siguiente análisis dimensional.
\begin{equation*}
    \frac
    {p_0 \, \gamma}{\rho_0}
    \equiv
    \frac{\dfrac{\si{\newton}}{\si{\metre^2}}}{\dfrac{\si{\kilo\gram}}{\si{\metre^3}}}
    =
    \frac{\dfrac{\si{\kilo\gram} \, \si{\metre}}{\si{\second^2}} \, \dfrac{1}{\si{\metre^2}}}{\dfrac{\si{\kilo\gram}}{\si{\metre^3}}} =
    \frac{\si{\metre^2}}{\si{\second^2}}
    \equiv
    c^2
\end{equation*}

% ¿Cómo sabés que esa velocidad es en efecto la velocidad de propagación de la onda?


\section{Ecuación de onda de presión en fluídos}

Consideremos una porción de cierto fluido dentro de un conducto de área transversal $\Delta y \, \Delta z$ con un pistón en un extremo.

\begin{center}
    \def\svgwidth{0.8\linewidth}
    \input{./images/pressure-wave-eqn.pdf_tex}
\end{center}

Eventualmente el fluido podría estar en movimiento pero se supone que está en reposo, de modo que $v_0=0\,\si{\metre\per\second}$ es nula si no hay perturbaciones.

En principio, el fluido está en equilibrio, contenido en un espacio volumétrico $V_0$ y tiene cierta presión inicial $p_0$ debido a la fuerza que hace contra el pistón por un lado y contra el resto del fluido por el otro.

Luego, el pistón hace una fuerza $F_1$ mayor a la necesaria para simplemente contener el fluido, generando una perturbación en el medio.
Las partículas del fluido pasan a estar más apretadas ya que disponen de un volumen $V_1$ menor al anterior, por lo que la presión $p_1$ va a ser mayor a la que tenía.
La porción de fluido se desplaza momentáneamente alejándose del punto de equilibrio.

La perturbación se propaga a lo largo del medio, pudiendo deducir lo que se conoce como ecuación \emph{de onda} que rige el comportamiento de la presión y el movimiento del fluido en el espacio y el tiempo.

\subsection{Ecuación de onda en 1D}

La capa de fluido que se encuentra contra el pistón es sometida a la fuerza impuesta aumentando la presión.
Pero como los fluidos son medios elásticos y compresibles, el movimiento de las partículas aledañas al pistón no afectan inmediatamente a las más lejanas.
La fuerza impuesta por el pistón afecta primero a las partículas cercanas, luego en menor medida a las partículas del centro, y finalmente a las partículas que están casi fuera de la porción de fluido considerada.
Con lo cual, se tiene un gradiente de presión a lo largo del eje $x$.
El incremento de fuerza es igual a menos el incremento de presión por el área del conducto:
\begin{equation*}
    \sum F = F_1 - F_0 = - \Delta p_1 \, \Delta y \, \Delta z
\end{equation*}

El incremento de la fuerza somete al fluido a una presión $p_1$ y la sumatoria de fuerzas según las leyes de Newton es:
\begin{align*}
    \sum F = F_1 - F_0  = m \, a &= \rho_1 \, V_1 \, \frac{\dif v_1}{\dif t}
    \\
    &= \rho_1 \, \Delta x \, \Delta y \, \Delta z \, \frac{\dif v_1}{\dif t}
    \\
    &= \Delta x \, \Delta y \, \Delta z \, \frac{\dif \left( \rho_1 \, v_1 \right)}{\dif t}
\end{align*}

Igualando:
\begin{equation}
    - \Delta p_1 \, \Delta y \, \Delta z
    =
    \Delta x \, \Delta y \, \Delta z \, \frac{\dif \left( \rho_1 \, v_1 \right)}{\dif t}
    \label{eqn:momentumDelta}
\end{equation}

Para una capa de fluido infinitamente delgada tal que $\Delta x \to 0$ se tiene:
\begin{equation}
    \frac{\dif p_1}{\dif x} = - \frac{\dif \left( \rho_1 \, v_1 \right)}{\dif t}
    \label{eqn:momentumDif}
\end{equation}

Si bien puede variar el volumen y la presión, la masa de la porción de fluido considerada se conserva.
Es la misma cantidad de partículas que ocupan más o menos lugar por estar menos o más apretadas respectivamente.
No obstante, al considerar una sección transversal fija en el espacio observamos que las partículas de fluido entran y salen de la misma, atravesándola.

Se analiza entonces el caudal másico dividido $\Delta x$:
\begin{align*}
    m &= \rho_1 V_1 = \rho_1 \, \Delta x \, \Delta y \, \Delta z
    \\
    \dot{m} &= \frac{\Delta m}{\Delta t}
    \\
    \frac{\dot{m}}{\Delta x} &= \frac{\Delta m}{\Delta t \, \Delta x}
    = \frac{\Delta \left( \rho_1 \, \Delta x \, \Delta y \, \Delta z \right)}{\Delta t \, \Delta x}
\end{align*}

A partir de la definición anterior, se comparan dos cocientes incrementales con respecto a diferentes variables independientes.
\begin{gather*}
    \left\{
    \begin{aligned}
        \frac{\dot{m}}{\Delta x} &= \frac{\Delta \left( \rho_1 \, \Delta y \, \Delta z \right)}{\Delta t}
        \\
        \\
        \frac{\dot{m}}{\Delta x} &= -\frac{\Delta \left( \rho_1 \, v_1 \, \Delta y \, \Delta z \right)}{\Delta x}
    \end{aligned}
    \right.
\end{gather*}

Observar que si la velocidad de las partículas del flujo es en el sentido de las $x$, la masa saliente es mayor a la entrante debido a la compresión, por lo que el caudal másico es negativo.

Al igualar las ecuaciones del sistema se tiene:
\begin{equation}
    \frac{\Delta \left( \rho_1 \, \Delta y \, \Delta z \right)}{\Delta t}
    =
    -\frac{\Delta \left( \rho_1 \, v_1 \, \Delta y \, \Delta z \right)}{\Delta x}
    \label{eqn:massDelta}
\end{equation}

Como el área $\Delta y \, \Delta z$ es constante, queda:
\begin{equation}
    \frac{\dif \rho_1}{\dif t} = - \frac{\dif \left( \rho_1 \, v_1 \right)}{\dif x}
    \label{eqn:massDif}
\end{equation}

Es necesario linealizar las ecuaciones \ref{eqn:momentumDif} y \ref{eqn:massDif} para poder usar las hipótesis de la acústica lineal (Sec. \ref{sec:linAcousticHyp}).

Teniendo en cuenta que:
\begin{equation*}
    \left\{
    \begin{aligned}
        p_1 &= p_0 + p(x,t)
        \\
        \rho_1 &= \rho_0 + \rho(x,t)
        \\
        v_1 &= v_0 + v(x,t) \quad \textrm{con } v_0=0\,\si{\metre\per\second}
    \end{aligned}
    \right.
\end{equation*}

La ecuación de la conservación del momento (Ec. \ref{eqn:momentumDif}) linealizada queda:
\begin{align*}
    \frac{\dif p_1}{\dif x}
    &= - \frac{\dif \left( \rho_1 \, v_1 \right)}{\dif t}
    \\
    \frac{\dif}{\dif x} \left[ p_0 + p(x,t) \right]
    &= - \frac{\dif}{\dif t} \left[ \rho_0 + \rho(x,t) \right] \left[ v_0 + v(x,t) \right]
    \\
    &= - \frac{\dif}{\dif t} \left[ \rho_0 \, v(x,t) + \rho(x,t) \, v(x,t) \right]
    \\
    \frac{\dif}{\dif x} p(x,t)
    &= - \rho_0 \, \frac{\dif}{\dif t} v(x,t) + \frac{\dif}{\dif t} \left[ \rho(x,t) \, v(x,t) \right]
\end{align*}

El término que tiene ambas variables multiplicadas tiende a cero con mayor orden y es despreciable, obteniendo la \emph{ecuación de Euler} para la conservación del momento:
\begin{equation}
    \frac{\dif}{\dif x} p(x,t) = - \rho_0 \, \frac{\dif}{\dif t} v(x,t)
    \label{eqn:EulerMomentumConservation}
\end{equation}

La ecuación de la conservación de la masa (Ec. \ref{eqn:massDif}) linealizada queda:
\begin{align*}
    \frac{\dif \rho_1}{\dif t} &= - \frac{\dif \left( \rho_1 \, v_1 \right)}{\dif x}
    \\
    \frac{\dif}{\dif t} \left[ \rho_0 + \rho(x,t) \right]
    &= - \frac{\dif}{\dif x} \left[ \rho_0 + \rho(x,t) \right] \left[ v_0 + v(x,t) \right]
    \\
    &= - \frac{\dif}{\dif x} \left[ \rho_0 \, v(x,t) + \rho(x,t) \, v(x,t) \right]
    \\
    \frac{\dif}{\dif t} \rho(x,t)
    &= - \rho_0 \, \frac{\dif}{\dif x} v(x,t) + \frac{\dif}{\dif x} \left[ \rho(x,t) \, v(x,t) \right]
\end{align*}

El término que tiene ambas variables multiplicadas tiende a cero con mayor orden y es despreciable:
\begin{equation}
    \frac{\dif}{\dif t} \rho(x,t) = - \rho_0 \, \frac{\dif}{\dif x} v(x,t)
    \label{eqn:EulerMassConservation}
\end{equation}

Se define el siguiente sistema de ecuaciones a partir de la ecuación de la conservación del momento linealizada (Ec. \ref{eqn:EulerMomentumConservation}), la ecuación de la conservación de la masa linealizada (Ec. \ref{eqn:EulerMassConservation}) y la hipótesis de la acústica lineal (Prop. \ref{prop:linearAcoustics}).
\begin{equation*}
    \left\{
    \begin{aligned}
        \frac{\dif}{\dif x} p(x,t) &= - \rho_0 \, \frac{\dif}{\dif t} v(x,t)
        \\
        \frac{\dif}{\dif t} \rho(x,t) &= - \rho_0 \, \frac{\dif}{\dif x} v(x,t)
        \\
        p(x,t) &= c^2\,\rho(x,t)
    \end{aligned}
    \right.
\end{equation*}

Derivando las dos primeras ecuaciones:
\begin{equation*}
    \left\{
    \begin{aligned}
        \frac{\dif^2}{\dif x^2} p(x,t) &= - \rho_0 \, \frac{\dif}{\dif x} \frac{\dif}{\dif t} v(x,t)
        \\
        \frac{\dif^2}{\dif t^2} \rho(x,t) &= - \rho_0 \, \frac{\dif}{\dif t} \frac{\dif}{\dif x} v(x,t)
        \\
        p(x,t) &= c^2\,\rho(x,t)
    \end{aligned}
    \right.
\end{equation*}

Y reemplazando la tercer ecuación en la segunda se infiere la ecuación de onda en una dimensión:
\begin{equation*}
    \left\{
    \begin{aligned}
        \frac{\dif^2}{\dif x^2} p(x,t) &= - \rho_0 \, \frac{\dif}{\dif x} \frac{\dif}{\dif t} v(x,t)
        \\
        \frac{\dif^2}{\dif t^2} \frac{p(x,t)}{c^2} &= - \rho_0 \, \frac{\dif}{\dif t} \frac{\dif}{\dif x} v(x,t)
    \end{aligned}
    \right.
\end{equation*}

\begin{mdframed}[style=DefinitionFrame]
    \begin{defn}
        \label{defn:soundWave1D}
    \end{defn}
    \cusTi{Ecuación de onda en 1D}
    \begin{equation*}
        \frac{\dif^2 p}{\dif t^2} - c^2 \, \frac{\dif^2 p}{\dif x^2} = 0
    \end{equation*}
\end{mdframed}


\subsection{Ecuación de onda en 3D}

El planteo en 3D es similar al caso de 1D, con la salvedad que la velocidad es una magnitud vectorial y que $\Delta y \, \Delta z$ no es constante.

A partir de la ecuación \ref{eqn:momentumDelta} se plantea la conservación del momento.
\begin{align*}
    - \Delta p_1 \, \Delta y \, \Delta z
    &=
    \Delta V \, \frac{\dif \left( \rho_1 \, \Vec{v}_1 \right)}{\dif t}
    \\
    - \iint_S p_1 \cdot \versor{n} \, \dif S
    &= \iiint_V \frac{\dif \left( \rho_1 \, \Vec{v}_1 \right)}{\dif t} \dif V
\end{align*}

Aplicando el teorema de Gauss a la integral de superficie del miembro izquierdo de la ecuación:
\begin{gather}
    - \iiint_V \grad \cdot p_1 \, \dif V
    =
    \iiint_V \frac{\dif \left( \rho_1 \, \Vec{v}_1 \right)}{\dif t} \dif V
    \notag\\
    \frac{\dif \left( \rho_1 \, \Vec{v}_1 \right)}{\dif t} + \grad \cdot p_1 = 0
    \label{eqn:momentum3D}
\end{gather}

% La presión es una magnitud escalar. La integral de superficie requiere de un campo vectorial.

Multiplicando por $\Delta x$ en ambos miembros de la ecuación \ref{eqn:massDelta} se plantea la conservación de la masa.
\begin{align*}
    \frac{\Delta \left( \rho_1 \, \Delta V \right)}{\Delta t}
    &=
    -\Delta \left( \rho_1 \, \Vec{v_1} \, \Delta y \, \Delta z \right)
    \\
    \iiint_V \frac{\dif \rho_1}{\dif t} \, \dif V
    &=
    - \iint_S \rho_1 \, \Vec{v}_1 \cdot \versor{n} \, \dif S
\end{align*}

Aplicando el teorema de Gauss a la integral de superficie del miembro derecho de la ecuación:
\begin{gather}
    \iiint_V \frac{\dif \rho_1}{\dif t} \, \dif V
    =
    - \iiint_V \grad \cdot \left( \rho_1 \, \Vec{v_1} \right) \dif V
    \notag\\
    \frac{\dif \rho_1}{\dif t} + \grad \cdot \left( \rho_1 \, \Vec{v_1} \right) = 0
    \label{eqn:mass3D}
\end{gather}

Las ecuaciones \ref{eqn:momentum3D} y \ref{eqn:mass3D} linealizadas quedan dadas respectivamente por:
\begin{equation*}
    \left\{
    \begin{aligned}
        \rho_0 \, \frac{\dif \Vec{v}_1}{\dif t} + \grad \cdot p &= 0
        \\
        \frac{\dif \rho}{\dif t} + \rho_0 \, \grad \cdot \Vec{v} &= 0
    \end{aligned}
    \right.
\end{equation*}

En el sistema anterior se considera una tercera ecuación, dada por la propiedad \ref{prop:linearAcoustics}, que surge de las hipótesis de la acústica lineal.

Siguiendo un procedimiento similar al caso unidimensional, se obtiene la ecuación de onda en tres dimensiones al despejar el sistema.

\begin{mdframed}[style=DefinitionFrame]
    \begin{defn}
        \label{defn:soundWave3D}
    \end{defn}
    \cusTi{Ecuación de onda en 3D}
    \begin{equation*}
        \ddot{p} - c^2 \, \grad^2 p = 0
    \end{equation*}
\end{mdframed}

A partir del laplaciano de presión para coordenadas esféricas surge la ecuación de onda tridimensional dada en función de la distancia $r$ a la fuente.

\begin{mdframed}[style=DefinitionFrame]
    \begin{defn}
        \label{defn:soundWaveSpherical}
    \end{defn}
    \cusTi{Ecuación de onda esférica}
    \begin{equation*}
        \frac{\partial^2 (rp)}{\partial t^2} - c^2 \frac{\partial^2 (rp)}{\partial r^2} = 0
    \end{equation*}
\end{mdframed}

\section{Soluciones a la ecuación de onda}

La ecuación de onda determina cómo varía la presión en el espacio y conforme transcurre el tiempo.
Las soluciones son, por lo tanto, campos escalares de presión $p:\setR^n\to\setR$ donde $n=2$ en el caso de ondas planas que se propaguen en una dimensión espacial y una temporal; O bien $n=4$ en el caso de ondas esféricas que se propaguen en tres dimensiones espaciales y una temporal.

La solución general a la ecuación de onda en una dimensión está dada por la suma de dos funciones cualesquiera cuyo argumento sea $t-\tfrac{x}{c}$ y $t+\tfrac{x}{c}$ respectivamente.
La función de argumento $t-\tfrac{x}{c}$ determina la propagación de la onda en el sentido positivo y la función de argumento $t+\tfrac{x}{c}$ lo hace en el sentido negativo.

\begin{mdframed}[style=DefinitionFrame]
    \begin{defn}
    \end{defn}
    \cusTi{Solución general en 1D}
    \begin{equation*}
        p(x,t) = f_1 \left( t - \frac{x}{c} \right) + f_2 \left( t + \frac{x}{c} \right)
    \end{equation*}
\end{mdframed}

Particularmente podemos representar ondas periódicas restringiendo la solución general funciones armónicas de la siguiente manera.

\begin{mdframed}[style=DefinitionFrame]
    \begin{defn}
        \label{defn:waveFunction1D}
    \end{defn}
    \cusTi{Solución periódica en 1D}
    \begin{equation*}
        p(x,t) = A \, e^{\iu \omega \left( t - \frac{x}{c}\right)} + B \, e^{\iu \omega \left( t + \frac{x}{c}\right)}
    \end{equation*}
\end{mdframed}

Una onda en tres dimensiones cuyo frente de onda se propague de forma esférica (Def. \ref{defn:soundWaveSpherical}) puede ser representada por una única variable espacial.
Las ondas esféricas solo dependen de la distancia $r$ a la fuente, ya que no hay variación de presión en las direcciones angulares.

\begin{mdframed}[style=DefinitionFrame]
    \begin{defn}
        \label{defn:sphericalWaveFunction}
    \end{defn}
    \cusTi{Solución periódica para onda esférica}
    \begin{equation*}
        p(r,t) = \frac{A}{r} \, e^{\iu \omega \left( t - \frac{r}{c}\right)} + \frac{B}{r} \, e^{\iu \omega \left( t + \frac{r}{c}\right)}
    \end{equation*}
\end{mdframed}

\section{Ecuación de Helmholtz}

\begin{mdframed}[style=DefinitionFrame]
    \begin{defn}
    \end{defn}
    \cusTi{Coeficiente de reflexión}
    \begin{equation*}
        R=\frac{B}{A}
    \end{equation*}
\end{mdframed}

Dado el número de onda $k=\tfrac{\omega}{c}$ podemos disociar la función de onda (Def. \ref{defn:waveFunction1D}) en espacio y tiempo.
\begin{align*}
    p(x,t) &= A \left[ e^{\iu \omega \left( t - \frac{x}{c}\right)} + R \, e^{\iu \omega \left( t + \frac{x}{c}\right)} \right]
    \\
    &= A \left( e^{\iu \omega t} \, e^{-\iu \frac{\omega}{c} x} + R \, e^{\iu \omega t} \, e^{\iu \frac{\omega}{c} x} \right)
    \\
    &= A \left( e^{-\iu \frac{\omega}{c} x} + R \, e^{\iu \frac{\omega}{c} x} \right) e^{\iu \omega t}
    \\
    &= A \left( e^{-\iu k x} + R \, e^{\iu k x} \right) e^{\iu \omega t}
\end{align*}

Sean:
\begin{gather}
    p(x) = A \left( e^{-\iu k x} + R \, e^{\iu k x} \right)
    \label{eqn:Helmholtz}
    \\
    p(t) = e^{\iu \omega t}
\end{gather}

Pudiendo escribir:
\begin{equation*}
    p(x,t) = p(x) \, p(t)
\end{equation*}

Y dado que $p(x,t)$ es solución de la ecuación de onda (Def. \ref{defn:soundWave1D}) entonces $p(x) \, p(t)$ también lo es:
\begin{gather*}
    \frac{\dif^2}{\dif x^2} \left[ p(x) \, p(t) \right] - \frac{1}{c^2} \, \frac{\dif^2}{\dif t^2} \left[ p(x) \, p(t) \right] = 0
    \\
    \frac{\dif^2 p(x)}{\dif x^2} \, p(t) - \frac{1}{c^2} \, \frac{\dif^2 p(t)}{\dif t^2} p(x) = 0
\end{gather*}

Luego, desarrollando la derivada:
\begin{equation*}
    \frac{\dif^2 p(t)}{\dif t^2} = \omega^2 \, \iu^2 \, e^{\iu \omega t} = - \omega^2 \, p(t)
\end{equation*}

Se obtiene:
\begin{equation*}
    \frac{\dif^2 p(x)}{\dif x^2} \, p(t) + \frac{\omega^2}{c^2} \, p(t) \, p(x) = 0
\end{equation*}

Y al dividir $p(t)$ se infiere:

\begin{mdframed}[style=DefinitionFrame]
    \begin{defn}
        \label{defn:Helmholtz1D}
    \end{defn}
    \cusTi{Ecuación de Helmholtz en 1D}
    \begin{equation*}
        \frac{\dif^2 p(x)}{\dif x^2} + k^2 \, p(x) = 0
    \end{equation*}
    Donde $k=\frac{\omega}{c}$ es el número de onda.
\end{mdframed}

Cuya solución está dada por la ecuación \ref{eqn:Helmholtz} definida anteriormente.

Se puede demostrar que la equivalente a la ecuación de Helmholtz en una dimensión está dada para las tres dimensiones cartesianas por:

\begin{mdframed}[style=DefinitionFrame]
    \begin{defn}
        \label{defn:Helmholtz3D}
    \end{defn}
    \cusTi{Ecuación de Helmholtz en 3D}
    \begin{equation*}
        \grad^2 p(\Vec{x}) + k^2 \, p(\Vec{x}) = 0
    \end{equation*}
\end{mdframed}

Y a su vez en coordenadas esféricas a partir del laplaciano de presión.

\begin{mdframed}[style=DefinitionFrame]
    \begin{defn}
        \label{defn:HelmholtzSpheric}
    \end{defn}
    \cusTi{Ecuación de Helmholtz en esféricas}
    \begin{equation*}
        \frac{\partial^2 (rp)}{\partial r^2} + k^2(rp) = 0
    \end{equation*}
\end{mdframed}

Cuya solución se obtiene al disociar en espacio y tiempo la función de onda esférica (Def. \ref{defn:sphericalWaveFunction}) siguiendo un razonamiento similar al caso unidimensional.
\begin{equation*}
    p(x) = \frac{A}{r} \left( e^{-\iu k x} + R \, e^{\iu k x} \right)
\end{equation*}


\section{Velocidad de partículas}

Así como la ecuación de Helmholtz y sus soluciones determinan la variación espacial de la presión independientemente del tiempo, se pretende definir la velocidad particular $v$ de igual manera.
\begin{gather*}
    v(x,t) = v(t) \, v(x) = e^{\iu \omega t} \, v(x)
    \\
    \frac{\partial}{\partial t} v(x,t) = \iu \omega \, e^{\iu \omega t} \, v(x)
\end{gather*}

Según la ecuación de Euler (Ec. \ref{eqn:EulerMomentumConservation}) se tiene que la derivada de la velocidad en función del tiempo es:
\begin{align*}
    \frac{\partial}{\partial t} v(x,t) &= - \frac{1}{\rho_0} \, \frac{\partial}{\partial x} p(x,t)
    \\
    &= - \frac{1}{\rho_0} \, \frac{\partial p(x)}{\partial x} \, p(t)
    \\
    &= - \frac{1}{\rho_0} \, \frac{\partial p(x)}{\partial x} \, e^{\iu \omega t}
\end{align*}

Y al igualar ambas derivadas se tiene:
\begin{align*}
    \iu \omega \, e^{\iu \omega t} \, v(x) &= - \frac{1}{\rho_0} \, \frac{\partial p(x)}{\partial x} \, e^{\iu \omega t}
    \\
    v(x) &= - \frac{1}{\iu \omega \rho_0} \, \frac{\partial p(x)}{\partial x}
\end{align*}

Obteniendo para $\omega=c\,k$ la siguiente expresión para la velocidad de las partículas en función de $x$:

\begin{mdframed}[style=DefinitionFrame]
    \begin{defn}
        \label{defn:particlesVelocity}
    \end{defn}
    \cusTi{Velocidad particular}
    \begin{equation*}
        v(x) = \frac{\iu}{\rho_0 \, c \, k} \, \frac{\partial p(x)}{\partial x}
    \end{equation*}
\end{mdframed}

\section{Velocidad de propagación y densidad}

La velocidad de propagación de la onda en un medio dado, se relaciona con la longitud de onda ($\lambda$), el número de onda ($k$), la frecuencia ($f$), el período ($T$) y la frecuencia angular ($\omega$) de la siguiente manera:
\begin{equation*}
    \lambda = \frac{2 \pi}{k} = \frac{2 \pi}{\omega/c} = \frac{c}{f} = c \, T
\end{equation*}

La velocidad de propagación del sonido depende de la temperatura del medio.
\begin{equation*} % ¿De dónde sale esto?
    c = 331 + 0.607 \, T
\end{equation*}

Con $T$ dado en $\si{\celsius}$

La densidad del medio en el que se propaga una onda de sonido puede ser calculada a partir de la ecuación general de los gases ideales.
\begin{gather*}
    p \, V = n \, R_G \, T
    \\
    p \, V = \frac{m}{M} \, R_G \, T
    \\
    \frac{m}{V} = \frac{p \, M}{R_G \, T}
\end{gather*}

De modo que la densidad del aire es:
\begin{equation*}
    \rho_0 = \frac{p_0 \, M}{R_G \, T} \approx 1.2\,\si{\kilo\gram\per\metre^3}
\end{equation*}

Donde:
\begin{align*}
    & p_0 = 1\,\si{\atmosphere} \quad\text{es la presión atmosférica.}
    \\
    & M = 28.9645\,\si{\kilo\gram\per\mol} \quad\text{es la masa molar del aire.}
    \\
    & R_G = 0.08206\,\si{\atmosphere\litre\per\mol\per\kelvin} \quad\text{es la cst. de los gases.}
    \\
    & T = (273.15+20)\,\si{\kelvin} \quad\text{es la temperatura ambiente.}
\end{align*}


\section{Impedancia acústica}

Según la definición \ref{defn:sound}, estudiar el sonido implica modelar matemáticamente la presión y la velocidad de las partículas como funciones periódicas.

La relación entre la presión $p(t)$ y la velocidad particular $v(t)$ es conocida como impedancia acústica.

\begin{mdframed}[style=DefinitionFrame]
    \begin{defn}
        \label{defn:impedance}
    \end{defn}
    \cusTi{Impedancia acústica}
    \cusTe{La impedancia acústica es la oposición a la propagación de una onda sonora en un medio.}
    \begin{equation*}
        Z \left(\Vec{x},t\right) = \frac{p \left(\Vec{x},t\right)}{v \left(\Vec{x},t\right)}
    \end{equation*}
\end{mdframed}


\subsection*{Impedancia en 1D}

La impedancia en una dimensión es:
\begin{equation*}
    Z(x) = \frac{p(x)}{v(x)}
\end{equation*}

Donde $p(x)$ es la presión, dada por la solución de la ecuación de Helmholtz en una dimensión (Def. \ref{defn:Helmholtz1D}):
\begin{equation*}
    p(x) = A \left( e^{-\iu k x} + R \, e^{\iu k x} \right)
\end{equation*}

Y $v(x)$ es la velocidad particular (Def. \ref{defn:particlesVelocity}):
\begin{align}
    v(x) &= \frac{\iu A}{\rho_0 \, c \, k} \, \left( -\iu k \, e^{-\iu k x} + \iu k \, R \, e^{\iu k x} \right)
    \notag
    \\
    &= \frac{A}{\rho_0 \, c} \left( e^{-\iu k x} - R \, e^{\iu k x} \right)
    \label{eqn:particleVel1D}
\end{align}

Obteniendo:

\begin{mdframed}[style=PropertyFrame]
    \begin{prop}
    \end{prop}
    \cusTi{Impedancia acústica en 1D}
    \cusTe{Si $0 \leq R \leq 1$ entonces:}
    \begin{equation*}
        Z(x) = \rho_0 \, c \, \frac{e^{-\iu k x} + R \, e^{\iu k x}}{e^{-\iu k x} - R \, e^{\iu k x}}
    \end{equation*}
    \cusTe{Si $R=1$ entonces:}
    \begin{equation*}
        Z(x) = \rho_0 \, c \, \frac{2 \cos(kx)}{-2 \iu \sin (kx)} = \iu \, \rho_0 \, c \, \tan^{-1}(kx)
    \end{equation*}
    \cusTe{Si $R=0$ entonces:}
    \begin{equation*}
        Z(x) = \rho_0 \, c
    \end{equation*}
\end{mdframed}


\subsection*{Impedancia en 3D}

Para una onda esférica en campo libre el coeficiente de reflexión es $R=0$.
Por lo tanto, la presión está dada según la ecuación de Helmholtz en coordenadas esféricas (Def. \ref{defn:HelmholtzSpheric}) por:
\begin{equation}
    p(r) = A \, \frac{e^{-\iu k r}}{r}
    \label{eqn:p(r)}
\end{equation}

Y la velocidad de las partículas está dada según la derivada de la presión (Def. \ref{defn:particlesVelocity}) como sigue:
\begin{align}
    v(r) &= \frac{\iu}{\rho_0 \, c \, k} \frac{\partial p(r)}{\partial r}
    \notag
    \\
    &= \frac{\iu \, A}{\rho_0 \, c \, k} \left( \frac{- \iu k}{r} \, e^{- \iu k r} - r^{-2} \, e^{- \iu k r} \right)
    \notag
    \\
    &= \frac{A}{\rho_0 \, c} \left( \frac{e^{- \iu k r}}{r} - \iu \, \frac{e^{- \iu k r}}{k \, r^2} \right)
    \notag
    \\
    &= \frac{A}{\rho_0 \, c} \, \frac{e^{- \iu k r}}{r} \left( 1 - \frac{\iu}{k \, r} \right)
    \label{eqn:v(r)}
\end{align}

Con lo cual, la impedancia acústica queda dada por:
\begin{align}
    Z(r) &= \frac{p(r)}{v(r)}
    \notag
    \\
    &= \frac{\rho_0 \, c}{1 - \dfrac{\iu}{k \, r}}
    \label{eqn:impedance3D}
\end{align}

Pudiendo separar la parte real de la imaginaria:
\begin{align*}
    Z(r) &= \frac{\rho_0 \, c}{1 - \frac{\iu}{k \, r}} \, \frac{1+\frac{\iu}{k \, r}}{1+\frac{\iu}{k \, r}}
    \\
    &= \frac{\rho_0 \, c \left( 1+\frac{\iu}{k \, r} \right)}{1^2-\left(\frac{\iu}{k \, r}\right)^2}
    \\
    &= \frac{\rho_0 \, c \left( 1+\frac{\iu}{k \, r} \right)}{1+\frac{1}{(k \, r)^2}}
    \\
    &= \rho_0 \, c \left[ \frac{1}{1+\frac{1}{(k \, r)^2}} + \iu \, \frac{\frac{1}{k \, r}}{1+\frac{1}{(k \, r)^2}} \right]
\end{align*}

O bien, multiplicando y dividiendo por $(k\,r)^2$ queda:

\begin{mdframed}[style=PropertyFrame]
    \begin{prop}
    \end{prop}
    \cusTi{Impedancia acústica en 3D}
    \begin{align*}
        Z(r) &= \Re(Z) + \iu \, \Im(Z)
        \\
        &= \rho_0 \, c \left[ \frac{(k \, r)^2}{(k \, r)^2+1} + \iu \, \frac{k \, r}{(k \, r)^2+1} \right]
    \end{align*}
\end{mdframed}

El término $k \, r$ determina si un punto a cierta distancia $r$ está en campo cercano o campo lejano, dependiendo de la frecuencia de la onda.
Al graficar $\Re(Z)$ y $\Im(Z)$ se puede observar que el número complejo $Z$ tiene mayor preponderancia en su componente real para valores de $k \, r$ altos, mientras que la componente imaginaria influye en mayor medida para valores de $k \, r$ bajos.

\begin{center}
    \def\svgwidth{0.8\linewidth}
    \input{./images/imped.pdf_tex}
\end{center}

A partir de la ecuación \ref{eqn:impedance3D} podemos hacer un análisis similar para el módulo de la impedancia.
Observar que $|Z|$ tiende a ser $\rho_0 \, c$ para valores de $k \, r$ altos.
\begin{equation*}
    \norm{Z(r)} = \frac{\rho_0 \, c}{\norm{1+\iu \, \dfrac{-1}{k \, r}}}
\end{equation*}

\begin{mdframed}[style=PropertyFrame]
    \begin{prop}
    \end{prop}
    \cusTi{Módulo de la impedancia en 3D}
    \begin{equation*}
        \norm{Z(r)} = \frac{\rho_0 \, c}{\sqrt{1+\dfrac{1}{(k \, r)^2}}}
    \end{equation*}
\end{mdframed}

\begin{center}
    \def\svgwidth{0.8\linewidth}
    \input{./images/imped-abs.pdf_tex}
\end{center}


\subsection{Efecto de proximidad}

El efecto de proximidad consiste en la disminución de la impedancia en torno a una fuente de sonido.
Se manifiesta por ejemplo en un realce de las frecuencias bajas captado por los micrófonos al posicionarlos cerca de una fuente.

Vemos que $Z \to 0$ cuando $v \to \infty$ independientemente de la presión.
Y esto se pone en evidencia tomando el valor eficaz para la presión y la velocidad de partículas de ondas esféricas, a partir de las ecuaciones \ref{eqn:p(r)} y \ref{eqn:v(r)} respectivamente.
\begin{align*}
    \rms{p}^2 &= \frac{\norm{p(r)}^2}{2}
    \\
    &= \frac{A^2}{2\,r^2}
\end{align*}
\begin{align*}
    \rms{v}^2 &= \frac{\norm{v(r)}^2}{2}
    \\
    &= \frac{A^2}{2\left(\rho_0 \, c \, r\right)^2} \left[ 1 + \frac{1}{(kr)^2} \right]
    \\
    &= \frac{A^2}{2\left(\rho_0 \, c \right)^2 r^2} + \frac{A^2}{2\left(\rho_0 \, c \right)^2 k^2 \, r^4}
\end{align*}

Para valores de frecuencia altos donde $k\to\infty$ el sumando en la velocidad se anula, pero para bajas frecuencias no es despreciable y para $r\to 0$ la velocidad de las partículas crece con mayor orden que la presión.


\section{Intensidad acústica}

La potencia es la cantidad de energía por unidad de tiempo.
En el caso de una fuente sonora, refiere a la energía radiada en forma de ondas sonoras.
Se define como sigue:

\begin{mdframed}[style=DefinitionFrame]
    \begin{defn}
    \end{defn}
    \cusTi{Potencia acústica}
    \cusTe{Flujo del campo vectorial de intensidad sobre una superficie Gaussiana que encierra la fuente sonora.}
    \begin{equation*}
        W = \iint_S \Vec{I} \cdot \dif\Vec{S}
    \end{equation*}
\end{mdframed}

Si la radiación es esférica, la intensidad es de módulo constante para cualquier punto de una porción de superficie.
La integral que queda, es el área de la porción de superficie.

En tal caso, la intensidad acústica se puede definir en función de la potencia de la fuente y la superficie de radiación:

\begin{mdframed}[style=DefinitionFrame]
    \begin{defn}
        \label{defn:I}
    \end{defn}
    \cusTi{Intensidad acústica}
    \cusTe{Potencia por unidad de superficie.}
    \begin{equation*}
        I = \frac{W}{S}
    \end{equation*}
\end{mdframed}

Partiendo de la definición \ref{defn:I}, se plantea para la intensidad acústica:
\begin{align*}
    \Delta I &= \frac{\Delta W}{S}
    \\
    &= \frac{\Delta \sub{E}{mec}}{S \, \Delta t}
    \\
    &= \frac{\Delta \sub{E}{mec}}{\Delta t \, V} \, \frac{V}{S}
    \\
    &= \frac{\Delta \sub{E}{mec}}{\Delta t \, V} \, \Delta x
    \\
    \frac{\Delta I}{\Delta x} &= \frac{\Delta \sub{E}{mec}}{\Delta t \, V}
\end{align*}

Tomando el límite cuando $\Delta t \to 0$ queda definida la densidad de energía mecánica $\rho_E=\sub{E}{mec}/V$ que es la energía mecánica por unidad de volumen:
\begin{equation}
    \frac{\dif I}{\dif x} = \frac{\dif \sub{E}{mec}}{\dif t \, V} = \frac{\dif \rho_E}{\dif t}
    \label{eqn:energyDensityI}
\end{equation}

A continuación se procede a calcular una equivalencia para la derivada de la densidad de energía, para deducir así la intensidad de manera indirecta.

Multiplicando por $v$ a ambos miembros de la ecuación de Euler (Ec. \ref{eqn:EulerMomentumConservation}) de la conservación del momento se obtiene:
\begin{equation*}
    \frac{\dif p}{\dif x} v = - \rho_0 \, \frac{\dif v}{\dif t} \, v
\end{equation*}

El miembro izquierdo de la ecuación anterior se puede expresar a partir de la derivada del producto $(pv)'=p'v+pv'$ al despejar $(pv)'-pv'=p'v$ quedando:
\begin{equation*}
    \frac{\dif (pv)}{\dif x} - p \, \frac{\dif v}{\dif x} = - \rho_0 \, \frac{\dif v}{\dif t} \, v
\end{equation*}

A partir de la ecuación \ref{eqn:EulerMassConservation} de la conservación de la masa y por la propiedad \ref{prop:linearAcoustics} de la acústica lineal, podemos expresar la derivada espacial de la velocidad de las partículas como $\tfrac{\dif v}{\dif x} = - \tfrac{1}{\rho_0 \, c^2} \tfrac{\dif p}{\dif t}$ de manera que:
\begin{equation*}
    \frac{\dif (pv)}{\dif x} + \frac{p}{\rho_0 \, c^2} \frac{\dif p}{\dif t} = - \rho_0 \, \frac{\dif v}{\dif t} \, v
\end{equation*}

Por regla de la cadena para cualquier función $f$ se tiene que $(f^2)'=2f \, f'$ por lo que en la ecuación anterior $p \, p'$ se puede expresar como $(p^2)'/2$ y análogamente $v \, v'$ como $(v^2)'/2$, obteniendo:
\begin{equation*}
    \frac{\dif (pv)}{\dif x} + \frac{1}{2 \, \rho_0 \, c^2} \frac{\dif p^2}{\dif t} = - \frac{\rho_0}{2} \, \frac{\dif v^2}{\dif t}
\end{equation*}

Al reagrupar los términos, queda definida la densidad de energía mecánica $\rho_E=\sub{E}{mec}/V$ según:
\begin{equation*}
    \frac{\dif (pv)}{\dif x} + \frac{\dif}{\dif t}  \underbrace{\left( \frac{p^2}{2 \, \rho_0 \, c^2} + \frac{\rho_0 \, v^2}{2} \right)}_{\rho_E}  = 0
\end{equation*}

Donde la energía cinética es $\sub{E}{cin} = \tfrac{\rho_0 \, v^2 \, V}{2}$ y la energía potencial $\sub{E}{pot} = \int_{V_0}^V p \, \dif V = \tfrac{p^2 \, V}{2 \, \rho_0 \, c^2}$ de modo que:
\begin{equation}
    \frac{\dif (pv)}{\dif x} + \frac{\dif \rho_E}{\dif t} = 0
    \label{eqn:energyDensityPv}
\end{equation}

% ¡Falta un signo menos!

Reemplazando la ecuación \ref{eqn:energyDensityPv} en la ecuación \ref{eqn:energyDensityI} se obtiene, finalmente:
\begin{align*}
    \frac{\dif I}{\dif x} &= \frac{\dif (pv)}{\dif x}
    \\
    \dif I &= \dif (pv)
    \\
    \int \dif I &= \int \dif (pv)
    \\
    I &= pv
\end{align*}

% ¿Cómo pasa de ser un escalar a ser vector?

\begin{mdframed}[style=DefinitionFrame]
    \begin{defn}
    \end{defn}
    \cusTi{Vector intensidad}
    \begin{equation*}
        \Vec{i}(x,t) = p(x,t) \, \Vec{v}(x,t)
    \end{equation*}
\end{mdframed}

% Hay un error en la página 108 del apunte, i(t) no debería estar al cuadrado.

Donde $\norm{\Vec{i}(x,t)}=i(x,t)$ puede ser escrito como:
\begin{align*}
    i(x,t) &= p(x,t) \, v(x,t)
    \\
    &= \Re \left[ p(x) \, e^{\iu \omega t} \right] \, \Re \left[ v(x) \, e^{\iu \omega t} \right]
\end{align*}

O bien, aplicando la propiedad $2\,\Re(z)=z+\overline{z}$, como:
\begin{align*}
    &= \frac{p(x) \, e^{\iu \omega t} + \overline{p}(x) \, e^{-\iu \omega t}}{2} \, \frac{v(x) \, e^{\iu \omega t}+ \overline{v}(x) \, e^{-\iu \omega t}}{2}
    \\
    &= \frac{p(x)\,v(x)\,e^{2\iu\omega t}}{4} 
    + \frac{p(x)\,\overline{v}(x)}{4}
    + \frac{\overline{p}(x)\,v(x)}{4}
    + \frac{\overline{p}(x)\,\overline{v}(x)}{4\,e^{2\iu\omega t}}
\end{align*}

La intensidad media está definida por la integral:
\begin{equation*}
    \ave{I} = \frac{1}{T} \int_0^T i(x,t)\,\dif t
\end{equation*}

Al integrar $i(x,t)$ en un período, se anulan el primer y cuarto sumando:
\begin{equation*}
    \ave{I} = \frac{1}{T} \int_0^T \frac{p(x)\,\overline{v}(x)+\overline{p}(x)\,v(x)}{4} \,\dif t
\end{equation*}

Haciendo uso de la propiedad $z\,\overline{w}=\overline{\overline{z}\,w}$, se tiene:
\begin{equation*}
    \ave{I} = \frac{1}{T} \int_0^T \frac{p(x)\,\overline{v}(x)+\overline{p(x)\,\overline{v}(x)}}{4} \,\dif t
\end{equation*}

Y usando nuevamente $z+\overline{z}=2\,\Re(z)$ se tiene:
\begin{align*}
    \ave{I} &= \frac{1}{T} \int_0^T \frac{2 \, \Re \left[ p(x)\,\overline{v}(x) \right]}{4} \,\dif t
    \\
    &= \frac{\Re \left[ p(x)\,\overline{v}(x) \right]}{2T} \underbrace{\int_0^T \dif t}_{=T}
\end{align*}

Obteniendo así la siguiente definición para la intensidad que puede ser extrapolada a notación vectorial para la velocidad.

\begin{mdframed}[style=PropertyFrame]
    \begin{prop}
        \label{prop:aveI}
    \end{prop}
    \cusTi{Intensidad media}
    \begin{equation*}
        \ave{I} = \frac{\Re \left[ p(x)\,\overline{v}(x) \right]}{2}
    \end{equation*}
\end{mdframed}

\section{Directividad} % Falta terminar

La restricción espacial que afecta la radiación esférica de la fuente se conoce como coeficiente de directividad $Q$.
En la figura a continuación sea muestran los valores que tomaría $Q$ para una fuente esférica ubicada en una esquina, al pié de una pared o contra el piso.

\begin{center}
    \def\svgwidth{0.6\linewidth}
    \input{./images/q.pdf_tex}
\end{center}

\begin{mdframed}[style=DefinitionFrame]
    \begin{defn}
    \end{defn}
    \cusTi{Índice de directividad}
    \begin{equation*}
        D_I = 10 \log (Q)
    \end{equation*}
\end{mdframed}


\section{Modelo de esfera pulsante}

Una fuente pulsante monopolar impone una velocidad de partículas en dirección radial, generando un frente de onda esférico.

\begin{mdframed}[style=DefinitionFrame]
    \begin{defn}
        \label{defn:Q_v}
    \end{defn}
    \cusTi{Caudal de velocidad particular}
    \begin{equation*}
        Q_v = S\,v(r)
    \end{equation*}
\end{mdframed}

El caudal de velocidad $Q_v$ que produzca una esfera pulsante de radio $a$ es:
\begin{equation*}
    Q_v = 4\pi a^2 \, v(a)
\end{equation*}

Donde $v(r)$ es la velocidad de partículas que impone la esfera pulsante sobre su superficie.
Evaluando $r=a$ en la ecuación \ref{eqn:v(r)} podemos escribir el caudal de velocidad como:
\begin{equation*}
    Q_v = 4\pi a^2 \, \frac{A}{\rho_0 \, c} \, \frac{e^{- \iu k a}}{a} \left( 1 - \frac{\iu}{k \, a} \right)
\end{equation*}

Resolviendo para $A$ se tiene:
\begin{align*}
    A &=  \frac{Q_v \, \rho_0 \, c}{4\pi a} \, \frac{e^{\iu k a}}{1 - \frac{\iu}{k \, a}}
    \\
    &= \frac{Q_v \, \rho_0 \, c}{4\pi a} \, \frac{e^{\iu k a}}{\tfrac{\iu k a + 1}{\iu k a}}
    \\
    &= \frac{Q_v \, \rho_0 \, c}{4\pi a} \, \frac{e^{\iu k a}}{\iu k a + 1} \, \frac{\iu \omega a}{c}
    \\
    &= \frac{\iu \omega \, Q_v \, \rho_0 }{4\pi} \, \frac{e^{\iu k a}}{1+\iu ka}
\end{align*}

Reemplazar esta amplitud calculada en la ecuación \ref{eqn:p(r)} permite definir la presión que genera la fuente en función de la distancia $r$ al centro de la esfera.
\begin{equation*}
    p(r) = \frac{\iu \omega \, Q_v \, \rho_0 }{4\pi} \, \frac{1}{1+\iu k a} \, \frac{e^{-\iu k (r-a)}}{r}
\end{equation*}

Considerar como hipótesis que $ka<<1$ supone que la fuente sea puntual, obteniendo para la presión:
\begin{equation}
    p(r) = \frac{\iu \omega \, Q_v \, \rho_0 }{4\pi} \, \frac{e^{-\iu k (r-a)}}{r}
    \label{eqn:p(r)withA}
\end{equation}

A partir de la definición \ref{defn:particlesVelocity}, al derivar la presión obtenida anteriormente, podemos determinar la velocidad de partículas que genera la fuente en función de la distancia $r$ al centro de la esfera.
\begin{align*}
    v(r) &= \frac{\iu}{\rho_0 \, c \, k } \, \frac{\dif p(r)}{\dif r}
    \\
    &= \frac{\iu}{\rho_0 \, c \, k } \, \frac{\iu \omega \, Q_v \, \rho_0 }{4\pi} \, \frac{\dif}{\dif r} \left[ \frac{e^{-\iu k (r-a)}}{r} \right]
    \\
    &= - \frac{Q_v}{4\pi} \, \frac{\dif}{\dif r} \left[ \frac{e^{-\iu k (r-a)}}{r} \right]
    \\
    &= - \frac{Q_v}{4\pi} \left[ -\iu k \frac{e^{-\iu k (r-a)}}{r} - \frac{e^{-\iu k (r-a)}}{r^2} \right]
    \\
    &= \frac{Q_v}{4\pi} \, \frac{e^{-\iu k (r-a)}}{r} \left( \frac{1}{r} + \iu k \right)
\end{align*}

Y su conjugado:
\begin{equation*}
    \overline{v}(r) = \frac{Q_v}{4\pi} \, \frac{e^{\iu k (r-a)}}{r} \left( \frac{1}{r} - \iu k \right)
\end{equation*}

De manera que el producto $p(r)\,\overline{v}(r)$ queda:
\begin{align*}
    p(r)\,\overline{v}(r) &= \frac{\iu \omega \, Q_v \, \rho_0 }{4\pi} \, \frac{e^{-\iu k (r-a)}}{r} \, \frac{Q_v}{4\pi} \, \frac{e^{\iu k (r-a)}}{r} \left( \frac{1}{r} - \iu k \right)
    \\
    &= \left(\frac{Q_v}{4\pi r}\right)^2 \left( \frac{\iu \, \omega \, \rho_0}{r} + \omega \, \rho_0 \, k \right)
    \\
    &= \left(\frac{Q_v}{4\pi r}\right)^2 \left( \frac{\omega^2 \, \rho_0}{c} + \iu \, \frac{\omega \, \rho_0}{r} \right)
\end{align*}

Tomando la parte real del producto anterior se obtiene, a partir de la propiedad \ref{prop:aveI}, la siguiente expresión para la intensidad:
\begin{align*}
    I &= \frac{\Re \left[ p(r)\,\overline{v}(r) \right]}{2}
    \\
    &= \left(\frac{Q_v}{4\pi r}\right)^2 \frac{\omega^2 \, \rho_0}{2c}
\end{align*}

La intensidad fue definida como la potencia por unidad de superficie (Def. \ref{defn:I}) luego $W=I\,S=I\,4\pi r^2$ pudiendo definir la potencia acústica al reemplazar la intensidad según:

\begin{mdframed}[style=PropertyFrame]
    \begin{prop}
    \end{prop}
    \cusTi{Potencia de fuente esférica puntual}
    \begin{equation*}
        W = \frac{Q_v^2 \, \rho_0 \, \omega^2}{8\pi \,c}
    \end{equation*}
\end{mdframed}

Por otro lado, podemos tomar la presión eficaz al cuadrado a partir de la ecuación \ref{eqn:p(r)withA} obteniendo:
\begin{align*}
    \rms{p}^2 &= \frac{\norm{p(r)}^2}{2}
    = \frac{1}{2} \, \norm{\frac{\iu \omega \, Q_v \, \rho_0 }{4\pi} \, \frac{e^{-\iu k (r-a)}}{r}}^2
    \\
    &= \frac{1}{2} \, \left(\frac{\omega \, Q_v \, \rho_0 }{4\pi r}\right)^2
    = \frac{1}{2} \, \frac{Q_v^2 \, \rho_0 \, \omega^2}{4\pi} \, \frac{\rho_0}{4\pi r^2}
    \\
    &= \frac{Q_v^2 \, \rho_0 \, \omega^2}{8\pi \, c} \, \frac{\rho_0 \, c}{4\pi r^2}
\end{align*}

Es necesario agregar como hipótesis que $k\,r>>1$ para considerar que la radiación se da en campo lejano.
Así, en la ecuación anterior, la impedancia está dada por $Z=\rho_0 \, c$ en efecto.
Luego:

\begin{mdframed}[style=PropertyFrame]
    \begin{prop}
        \label{prop:p^2/Z=W/S}
    \end{prop}
    Se tiene una fuente de sonido esférica, cuyo coeficiente de reflexión es $R=0$ y coeficiente de directividad $Q=1$.
    Sea $k\, a<<1$ pudiendo considerarse la fuente puntual.
    Sea $k\,r>>1$ pudiendo considerarse la radiación en campo lejano.
    Entonces se cumple:
    \begin{equation*}
        \frac{\rms{p}^2}{Z\,Q} = \frac{W}{S}
    \end{equation*}
    Donde $S=4\pi r^2$ es la superficie de radiación y $Z=\rho_0 \, c$ la impedancia característica del medio.
\end{mdframed}

Observar que cumplirse simultáneamente $k\, a<<1$ y $k\,r>>1$ implica que $r>>a$ lo cual significa que la propiedad se verifica para todas las frecuecias, por ser independiente de $k$.


\subsection{Ley cuadrática inversa}
\label{sec:inverseSquareLaw}

Haciendo algunos despejes en la propiedad \ref{prop:p^2/Z=W/S} se puede hallar una importante simetría.
\begin{align*}
    \frac{\rms{p}^2}{\rho_0 \, c} &= \frac{W}{4\pi r^2}
    \\
    \norm{\rms{p} \, r} &= \sqrt{\frac{W \, \rho_0 \, c}{4\pi}} = \textrm{constante}
\end{align*}

Lo cual implica:

\begin{mdframed}[style=PropertyFrame]
    \begin{prop}
    \end{prop}
    La presión multiplicada por la distancia es constante.
    \begin{equation*}
        \Delta \norm{\rms{p} \, r} = 0
    \end{equation*}
\end{mdframed}

Pudiendo escribir:
\begin{align*}
    {\rms{p}}_1 \, r_1 &= {\rms{p}}_2 \, r_2
    \\
    \frac{{\rms{p}}_1}{{\rms{p}}_2} &= \frac{r_2}{r_1}
    \\
    \frac{\tfrac{{\rms{p}}_1}{\sub{p}{ref}}}{\tfrac{{\rms{p}}_2}{\sub{p}{ref}}} &= \frac{r_2}{r_1}
    \\
    20 \log \left( \frac{{\rms{p}}_1}{\sub{p}{ref}} \right) - 20 \log \left( \frac{{\rms{p}}_2}{\sub{p}{ref}} \right) &= 20 \log \left( \frac{r_2}{r_1} \right)
    \\
    \Delta L_p = 20 \log \left( \frac{r_2}{r_1} \right)
\end{align*}

Luego si la distancia $r_2$ es el doble de $r_1$ queda:
\begin{equation*}
    \Delta L_p = 20 \log \left( \frac{2 \, r_1}{r_1} \right) \approx 6\,\si{\deci\bel}
\end{equation*}


%\section{Dipolo acústico} % Falta terminar (ver unidad 4)


%\section{Bafle infinito} % Falta terminar (ver unidad 4)


%\section{Modelo de Line Array} % Falta terminar (ver unidad 4)


\section{Niveles de presión, intensidad y potencia}

Para estudiar las variaciones de presión, intensidad y potencia, se definen niveles en escalas logarítmicas.
Los respectivos niveles relacionan dichas magnitudes con un valor de referencia, indicando qué tan por encima están del umbral de audición.

\begin{mdframed}[style=DefinitionFrame]
    \begin{defn}
    \end{defn}
    \cusTi{Presión de referencia}
    \cusTe{Mínima presión perceptible por el oído humano.}
    \begin{equation*}
        \sub{p}{ref} = 20\,\si{\micro\pascal}
    \end{equation*}
\end{mdframed}

Reemplazando $\sub{p}{ref}$ en la propiedad \ref{prop:p^2/Z=W/S} se obtiene la intensidad de referencia en el aire para ondas planas en campo lejano:
\begin{equation*}
    \sub{I}{ref}=\frac{(20\,\si{\micro\pascal})^2}{1.18\,\si{\kilo\gram\per\metre^2} \cdot 343\,\si{\metre\per\second}} = 10^{-12}\,\si{\watt\per\metre^2}
\end{equation*}

Reemplazando $\sub{I}{ref}$ en la definición \ref{defn:I} se tiene que la mínima potencia que puede percibir el humano, para una superficie de radiación $S=1\,\si{\metre^2}$, es:
\begin{equation*}
    \sub{W}{ref} = 10^{-12}\,\si{\watt}
\end{equation*}

\begin{mdframed}[style=DefinitionFrame]
    \begin{defn}
        \label{defn:SPL}
    \end{defn}
    \cusTi{Nivel de presión sonora}
    \begin{equation*}
        L_p = SPL = 20 \log \left( \frac{\rms{p}}{\sub{p}{ref}} \right) \si{\deci\bel}
    \end{equation*}
\end{mdframed}

\begin{mdframed}[style=DefinitionFrame]
    \begin{defn}
        \label{defn:SIL}
    \end{defn}
    \cusTi{Nivel de intensidad sonora}
    \begin{equation*}
        L_I = SIL = 10 \log \left( \frac{I}{\sub{I}{ref}} \right)\si{\deci\bel}
    \end{equation*}
\end{mdframed}

\begin{mdframed}[style=DefinitionFrame]
    \begin{defn}
        \label{defn:PWL}
    \end{defn}
    \cusTi{Nivel de potencia acústica}
    \begin{equation*}
        L_W = PWL = 10 \log \left( \frac{W}{\sub{W}{ref}} \right)\si{\deci\bel}
    \end{equation*}
\end{mdframed}

Para ondas planas en campo libre se puede usar la propiedad \ref{prop:p^2/Z=W/S} en la definición \ref{defn:SIL}, quedando:
\begin{equation*}
    SIL = 10 \log \left( \frac{\rms{p}^2}{\rho_0 \, c \, \sub{I}{ref}} \right) \si{\deci\bel}
\end{equation*}

Se multiplica el argumento del logaritmo por $\tfrac{\sub{p}{ref}^2}{\sub{p}{ref}^2}$ sin alterar la igualdad, y luego se reagrupa usando propiedades logarítmicas:
\begin{align*}
    SIL &= 10 \log \left( \frac{\rms{p}^2}{\sub{p}{ref}^2} \frac{\sub{p}{ref}^2}{\rho_0 \, c \, \sub{I}{ref}} \right) \si{\deci\bel}
    \\
    &= 20 \log \left( \frac{\rms{p}}{\sub{p}{ref}} \right) \si{\deci\bel} + 10 \log \left( \frac{\sub{p}{ref}^2}{\rho_0 \, c \, \sub{I}{ref}} \right) \si{\deci\bel}
\end{align*}

Verificándose la siguiente propiedad.

\begin{mdframed}[style=PropertyFrame]
    \begin{prop}
    \end{prop}
    Para ondas planas y en campo libre:
    \begin{equation*}
        SIL = SPL + 10 \log \left( \frac{\sub{p}{ref}^2}{\rho_0 \, c \, \sub{I}{ref}} \right) \si{\deci\bel}
    \end{equation*}
\end{mdframed}

En la que el segundo término tiende a anularse si los valores de $\sub{I}{ref}$, $\sub{p}{ref}$, $\rho_0$ y $c$ están dados para condiciones normales de presión y temperatura, quedando:
\begin{equation*}
    SIL \approx SPL
\end{equation*}

Despejando $\rms{p}^2$ de la propiedad \ref{prop:p^2/Z=W/S}, y reemplazando luego este valor en la definición de nivel de presión sonora (Def. \ref{defn:SPL}) se tiene:
\begin{align*}
    L_p &= 10 \log \left( \frac{W \, \rho_0 \, c \, Q}{S \, \sub{p}{ref}^2} \right) \si{\deci\bel}
    \\
    &= \left[ 10\log \left( \frac{\rho_0 \, c}{\sub{p}{ref}^2} \right) + 10\log(W) + 10\log \left( \frac{Q}{S} \right) \right] \si{\deci\bel}
    \\
    &= 120 \si{\deci\bel} + 10 \log(W)\,\si{\deci\bel} + 10 \log \left( \frac{Q}{S} \right) \si{\deci\bel}
\end{align*}

Aplicando la definición de nivel de potencia (Def. \ref{defn:PWL}) al sumando $120\,\si{\deci\bel}$ se tiene:
\begin{align*}
    L_p &= \left[ 10 \log \left( \frac{1}{10^{-12}} \right) + 10 \log(W) + 10 \log \left( \frac{Q}{S} \right) \right] \si{\deci\bel}
    \\
    &= 10 \log \left( \frac{W}{10^{-12}} \right) + 10 \log \left( \frac{Q}{S} \right) \si{\deci\bel}
\end{align*}

Obteniendo la siguiente propiedad.

\begin{mdframed}[style=PropertyFrame]
    \begin{prop}
    \end{prop}
    Para ondas planas y en campo libre:
    \begin{equation*}
        L_p = L_W + 10 \log \left( \frac{Q}{S} \right) \si{\deci\bel}
    \end{equation*}
\end{mdframed}

De manera que si la superficie es unitaria $S=1\,\si{\metre^2}$ se tiene que:
\begin{equation*}
    PWL \approx SPL \approx SIL
\end{equation*}